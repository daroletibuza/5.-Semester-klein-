\section{Fehlerbetrachtung}
\label{sec:fehler}
Für Fehler wird hauptsächlich die Messreihe 1 mit \SI{80}{\per\minute} betrachtet. Da sich in den Leitungen für die Sauerstoffzufuhr noch Stickstoff vom Stripping-Prozess befand kam es zur Beeinträchtigung der Messwerte. Daraufhin wurde die Sauerstoffzufuhr nachgeregelt.  Aufgrund dieser Tatsache ist es empfehlenswert die erste Messreihe zu wiederholen.
Da die Messreihen 2 und 3 einen ähnlichen, jedoch nicht gleichen Verlauf der Messdaten anzeigen und als plausibel gewertet werden können, kann hier auf eine Wiederholung der Versuchsdurchführung verzichtet werden.\\

Ansonsten lassen sich die getroffenen Annahmen unter Abschnitt \ref{sec:ergebnisse} eventuell anzweifeln, welche zu Anfang der Auswertung der Messdaten getroffen wurden. Zusätzlich dazu unterliegen Messeinrichtungen immer gewissen Schwankungen, sowie Fehlertoleranzen. Diese sind jedoch weder im Praktikum noch in der Auswertung negativ aufgefallen.
