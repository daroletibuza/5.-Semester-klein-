\section{Ergebnisse}
\label{sec:ergebnisse}

Mit den eingewogenen \SI{1,29}{\gram} 2-Naphthol und der Probenmasse von \SI{2,97}{\gram} berechnet sich in Gleichung \ref{gl:ausbeute} die Ausbeute für diese Versuchsdurchführung.
\begin{flalign}
	\label{gl:ausbeute}
	\eta &= \frac{m_{real}}{m_{theo}}= \frac{m_{real}}{m_{\text{2-Naphthol}*\frac{M_{\text{Orange II}}}{M_{\text{2-Naphthol}}}}}\\
	&= \frac{\SI{2,97}{\gram}}{\SI{1,29}{\gram}*\frac{\SI{350,32}{\gram \per \mol}}{\SI{144,17}{\gram \per \mol}}}\\[2mm]
	&= \underline{\SI{94,75}{\percent} \approx \SI{95}{\percent}}
\end{flalign}

\vspace*{5mm}

Nach Gleichung \ref{gl:rf} ergeben sich für die $R_f$-Werte der Proben damit:
\begin{flalign}
	\label{gl:rf}
	R_f  &= \frac{\text{Laufstrecke der Substanz}}{\text{Laufstrecke des Lösungsmittels}}
\end{flalign}
\begin{flalign}
	R_f (\text{Orange II}) &= \frac{\SI{3,0}{\centi\meter}}{\SI{7,4}{\centi \meter}}=\underline{0,411}
\end{flalign}
\begin{flalign}
	R_f (\text{2-Naphthol}) &= \frac{\SI{6,8}{\centi\meter}}{\SI{7,4}{\centi \meter}}=\underline{0,918}
\end{flalign}
