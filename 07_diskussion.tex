\newpage
\section{Diskussion}
\label{sec:diskussion}
Aus den Ergebnissen des Praktikumsversuches und den Berechnungen lässt sich zunächst erkennen, dass sich die Messreihe 1 als nicht plausibel herausstellt. Dies lässt sich auf Durchführungsfehler während des Praktikums zurückführen (siehe Abschnitt \ref{sec:fehler}). Die Messreihen 2 und 3 zeigen jedoch verwertbare Daten, aus denen mögliche Schlüsse gezogen werden können.\\

Zunächst zeigt sich in Abb. \ref{dia:konzentration}, dass im gleichen Zeitintervall Messreihe 3 mit \SI{140}{\per \minute} eine höhere Sauerstoffkonzentration erreicht, als Messreihe 2 mit \SI{110}{\per \minute}. Als These lässt sich hierbei formulieren, dass bei höheren Drehzahlen ein besserer Stoffübergang erfolgt.\\

Auch in Bezug auf die Stoffmengenänderungsgeschwindigkeiten zeigt Messreihe 3 eine effizientere Performance als Messreihe 2. Zu sehen ist dies an dem steileren Abfall der jeweiligen Regressionsgeraden (siehe Abb. \ref{dia:stoffmengenaenderung}), welcher einem schnelleren Umsatz entspricht.\\

Betrachtet man Abbildung \ref{dia:reynolds} so werden Unterschiede in der differentiellen und in der integrativen Methode deutlich. Vorrangig lässt sich jedoch der integrative Wert kritisch bewerten, da dafür die Annahme der Phasenkonzentration $c^\ast$ lediglich auf dem letzten Messwert der jeweiligen Messreihe basiert. Dies kann zu entscheidenden Ungenauigkeiten führen. Somit werden auch die folgenden Werte, die aus dieser Grafik mathematisch zu entnehmen sind auf der differentiellen Methodik basieren. \\
Ob man nun differentiell oder integrativ die Graphen betrachtet ist jedoch unwesentlich, sobald man den generellen Zusammenhang zwischen Reynoldszahl und dem Stoffübergangskoeffizienten betrachtet. Es fällt auf, dass auch hier bei einer höheren Reynoldszahl, sprich einer höheren Drehzahl, ein höherer Stoffübergangskoeffizient auftritt. Zu erkennen ist die dies an den geringeren Werten für $\ln\left(k_l*a\right)$.
Die ursprüngliche These, dass bei höheren Drehzahlen der Stoffübergang gefördert wird, wird somit nochmals untermauert.\\

Um nun die Messreihen untereinander nochmals auf Plausibilität zu prüfen, werden die berechneten \textsc{Henry}-Konstanten mit Literaturwerten verglichen.\\
\vspace*{-5mm}
% Table generated by Excel2LaTeX from sheet 'Daten'
\begin{table}[h!]
	\renewcommand*{\arraystretch}{1.2}
	\centering
	\rowcolors{2}{white}{gray!25}
	\caption{Vergleich der \textsc{Henry}-Konstanten}
	\label{tab:henry}
	\resizebox{14cm}{!}{
		\begin{tabulary}{1.15\textwidth}{C|CCC|C}
			\hline
			 & \SI{80}{\per \minute} & \SI{110}{\per \minute}&\SI{140}{\per \minute}&\textbf{Literatur} \cite[S. 535]{Draxler.2014}\\
			\hline
			\textsc{Henry}-Konstante ( für $T=\SI{20}{\celsius}$)&\SI{44937}{\bar}&\SI{40837}{\bar}&\SI{40751}{\bar}&\SI{38286}{\bar}\\
			Abweichung gegenüber Literaturwert& \SI{17,4}{\percent}&\SI{6,7}{\percent}&\SI{6,4}{\percent}&-\\
			Abweichung gegenüber \SI{140}{\per \minute}& \SI{10,3}{\percent}&\SI{0,2}{\percent}&\SI{0}{\percent}& -\\
			\hline			
	\end{tabulary}}
\end{table}%
\FloatBarrier

In Tabelle \ref{tab:henry} sind wiederholt die einzelnen \textsc{Henry}-Konstanten aufgeführt. Ebenfalls dargestellt sind die Abweichungen der Konstanten in Bezug auf Messreihe 3 und auf einen recherchierten Literaturwert.\\
Für die \textsc{Henry}-Konstanten wäre zu erwarten gewesen, dass diese für jede Messreihe demselben Zahlenwert entsprechen. Da dies aufgrund von Messtoleranzen und Annahmen kaum möglich ist, sollten die Werte auch mit einer gewissen Toleranz bewertet werden.\\
So fällt zunächst auf, dass sich alle Werte für die \textsc{Henry}-Konstante in einem ähnlichen Zahlenbereich befinden. Das größte Vertrauen wird hierbei dem Literaturwert zugeordnet, da dieser aus einem Fachbuch stammt \cite{Draxler.2014}. Vergleicht man die Abweichung der \textsc{Henry}-Konstanten mit dem Literaturwert so fällt wiederholt die Messreihe 1 auf. Sie besitzt mit \SI{17,4}{\percent } die höchste Abweichung gegenüber dem Literaturwert und bewegt sich außerhalb der Abweichungen der Messreihen 2 und 3. 
\newpage
Diese besitzen Abweichungen zwischen 6,0 - \SI{7}{\percent}. Somit bestätigt sich die fehlerhafte Durchführung der Messreihe auch in dieser Auswertung und die Messdaten gelten somit als nicht plausibel.\\
Betrachtet man die Messreihen 2 und 3 so könnte man behaupten, dass diese als plausibel gelten, wenn die Abweichrate mit 6-\SI{7}{\percent} für diesen Versuch nicht zu hoch erscheint. Für eine weitere Validierung der Messreihen 2 und 3 könnten weitere Literaturwerte hinzugezogen werden.\\
Vergleicht man die Konstanten der Messreihen nochmals untereinander in Bezug auf Messreihe 3, so zeigt sich das vor allem, dass die Messreihen 2 und 3 untereinander geringe Abweichungen im Vergleich zu Messreihe 1 und 3 haben. Somit würden Messreihe 2 und 3 auch die These der Konstanz der \textsc{Henry}-Konstante bestätigen und könnten somit als plausibel gewertet werden.\\

Nach diesen Ausführungen stellt sich nun die Frage, welche Schlussfolgerungen für den Stoffübergang von Sauerstoff in Wasser für diesen Versuch getroffen werden können.
Die These, dass mit erhöhter Drehzahl ein effizienterer Stofftransport gewährleistet wird, kann mit den Messreihen 2 und 3 in den vorangegangenen Ausführungen bestätigt werden. Eine höhere Drehzahl des Rührers verbessert für diesen Versuchsaufbau die Absorption von Sauerstoff in Wasser. \\
Der Stoffübergang könnte sich möglicher Weise mit einem geeigneteren Rührer, wie beispielsweise einem Scheibenrührer oder einem Rohrsegmentrührer verbessern, da diese auch in der Praxis für Begasungsprozesse Anwendung finden. Ein Ankerrührer ist für diese Anwendung eher ungeeignet.\\
Weiterhin könnte, wenn möglich, ein anders Stoffsystem genutzt werden, welches Sauerstoff besser absorbiert als Wasser und somit die Diffusion durch die Phasengrenzen erhöht. Ebenfalls könnte die Phasengrenzfläche erhöht werden, um mehr Stoffkontakt zu ermöglichen. So wäre es denkbar, dass die Luft mit noch feineren Bläschen in die Lösung eingedüst wird. Zuletzt bleibt noch die Möglichkeit die Hydrodynamik zu verbessern. Dies könnte in Form einer Verringerung der jeweiligen Dicken der  Phasengrenzen möglich sein.
