\newpage
\section{Diskussion der Ergebnisse}
\label{sec:diskussion}

In diesem Abschnitt des Protokolls werden die aufgenommenen Messreihen 1 bis 6 untereinander verglichen und ausgewertet.\\

Die in Tabelle \ref{tab:messdaten} aufgeführten Messdaten weisen zunächst keine Auffälligkeiten auf. Lediglich ein Teil der aufgenommen Messwerte ist unter Tabelle \ref{tab:mess_luft} zu finden, da Messreihe 6 mit einem luftgekoppelten, statt mit einem wasserdurchströmten Verdampfer aufgenommen wurde.

Beginnend mit dem mittleren Volumen- und Massenstrom fallen diese mit \SI{0,18}{\kmeter \per \second} und \SI{0,22}{\kg \per \per \second} als gering aus. Unter Einbezug der geometrischen Gegebenheiten des Verdampfers und der Luftgeschwindigkeiten erscheinen diese Werte jedoch als plausibel \linebreak (siehe Tab. \ref{tab:mess_luft}, Gl.  \ref{gl:volumenstrom}, Gl. \ref{gl:massenstrom}).\\

Weiter mit den Ausführungen in Tabelle \ref{tab:auswertung} erscheinen direkt mehrere Punkte von Interesse.\\
Zuerst fällt auf, dass die Kompressorleistung, berechnet mittels Kältemittelmassenstroms und den entsprechenden Kältemittelenthalpien, deutlich geringer ausfällt, als die aus der Messwertaufnahme aufgenommenen Leistungen (vgl. Tab. \ref{tab:messdaten}, Tab. \ref{tab:auswertung}). 
Als Grund hierfür wird vermutet, dass die gemessene Kompressorleistung, der elektrischen, aufzubringenden Leistung  entspricht und somit höher ausfällt. Da diese Leistung auch den damit zusammenhängenden Energieaufwand des Kompressors bemisst, wurden weitere Berechnungen und Betrachtungen auf die höhere, gemessene Kompressorleistung bezogen. Weitere Gründe finden sich unter Abschnitt \ref{sec:fehler}.\\
Entsprechend der Erwartung, dass mit steigendem Kältemittelstrom die aufzubringende Leistung des Verdampfers steigt, bestätigt sich dies innerhalb der Messreihen 1 bis 3. Bei diesen Messreihen ist lediglich der Kühlwasserstrom des Kompressors bzw. Kondensators
bei konstantem Kühlwassermassenstrom durch den wasserdurchströmten Verdampfer verändert worden.\\
In den Messreihen 4 und 5 hingegen wurden der Massenstrom des Kühlwasser für Kompressors bzw. Kondensators konstant gehalten und der Kühlwasserstrom des wasserdurchströmten Verdampfers verändert. Hier ist ebenfalls zu erkennen, dass eine Verringerung des veränderten Kühlwasserstromes eine Verringerung der Kompressorleistung, sowie des Kältemittelmassenstroms hervorruft.\\
Messreihe 6 zeigt hingegen, dass wenn Luft, statt Wasser zum Verdampfen genutzt wird ein deutlich geringerer Massenstrom an Luft genutzt wird. Die gemessene Kompressorleistung, sowie der Kühlwassermassenstrom an Kondensator bzw. Kompressor lassen sich jedoch mit Messreihe 4 vergleichen. \\

Im nächsten Zeilenabschnitt der Tabelle \ref{tab:auswertung} sind die abgelesenen Kältemittelenthalpien der Punkte 1 bis 4 des $h-\lg{P}$-Diagramms aufgeführt. Sie wurden für weitergehende Wärmeberechnungen genutzt. Angesichts der Größenordnung erscheinen diese Werte als plausibel.\\

Die aufgeführte Verdampfungstemperatur ist ein Messwert, welcher ebenfalls in Tabelle \ref{tab:messdaten} aufgeführt ist. Da R134a ein niedrig verdampfendes Kältemittel ist, erscheinen die negativen Temperaturen in Grad Celsius als realistisch und können aufgrund der direkten Messwertaufnahme als wahr angenommen werden.\\
Innerhalb der Messreihen 1 bis 3 zeigt sich, dass mit zunehmendem Kühlwassermassenstrom innerhalb der Messreihen 1 bis 3 die Verdampfungstemperatur steigt. Innerhalb der Messreihen 4 und 5 sinkt die Verdampfungstemperatur mit steigendem Kühlwasserstrom am Verdampfer. In Messreihe 6 fällt auf, dass diese Messreihe unter Nutzung des luftgekoppelten Verdampfers, die höchste Verdampfungstemperatur des Kältemittels aufweist.\\

Ebenfalls wie die Verdampfungstemperatur steigt auch die aus dem Diagramm bestimmte Kondensationstemperatur der Messreihen 1 bis 3 beim gemessenen Kondensatordruck. Innerhalb der Messreihen 4 und 5 sinkt diese ebenfalls, wie zuvor bei der Verdampfungstemperatur, jedoch im deutlich geringeren Ausmaß. Messreihe 6 zeigt sich mit der selben Kondensationstemperatur, wie Messreihe 5,  jedoch mit dem Unterschied, dass die zuvor betrachtete Verdampfungstemperatur in Messreihe 5 knapp \SI{6}{\kelvin} unter der Messreihe 6 des luftgekoppelten Verdampfers liegt.\\

Das spezifische Volumen des Kältemittels am Punkt 1 zeigt sich in den Messreihe 1 bis 3 mit steigendem Kühlwasserstrom an Kondensator bzw. Kompressor als fallend. Auch innerhalb der Messreihen 4 bis 5 ist die genau umgekehrte Tendenz der Messreihen 1 bis 3 im geringeren Maße zu erkennen. Jedoch zeigt Messreihe 6 einen deutlich erhöhten Wert des spezifischen Volumen an Kältemittel. Dieser ist mit \SI{78}{\kmeter \per \kg} im Vergleich zum höchsten Wert der Messreihen 1 bis 5 mit \SI{30}{\kmeter \per \kg} mer als doppelt so groß.
Der Trend der spezifischen Volumina zeigt sich analog auch im Volumenstrom durch den Kompressor.\\
Betrachtet man den Kältemittel Massenstrom so verläuft dieser mit weniger Abweichungen geläufig zu den Beobachtungen des spezifischen Volumens. Der Kältemittelstrom der Messreihe 6 ist jedoch nicht so auffällig, wie in den vorangegangenen Ausführungen.

Es folgt die Diskussion der übertragenen Wärmen.\\
Innerhalb des Verdampfers wird für die Messreihen 1 bis 5 Wärme vom Wasser und für die Messreihe 6, Wärme der Luft zum Verdampfen des Kältemittels übertragen. Zu erwarten ist für die Messreihen 1 bis 3, dass mit steigenden Kühlwasserstrom am Kompressor der Betrag der übertragenen Wärme an das Kältemittel steigt. Für die Messreihen 4 und 5 wird wiederholt ein geläufiger Trend erwartet. Es ist aufgrund von Wärmeverlusten damit zu rechnen, dass die vom Wasser/Luft berechnete, abgegebene Wärme größer ausfällt als die vom Kältemittel aufgenommene Wärme. Diese Vermutung wird durch die Messreihen 1, 3, 4 und 6 bestätigt, aber in den Messreihen 2 und 5 lässt sich eine gegenteilige Meinung begründen. Ein Grund hierfür könnte sich in den Kältemittelenthalpien finden, welche als nicht ausreichend genau abgelesen sein könnten. Weiterhin ist eine Wärmeaufnahme des Kühlmittels durch die Umgebung selbst nicht ausgeschlossen. Dieser Fehler müsste sich jedoch auf alle Messreihen gleichermaßen auswirken.\\
Noch auffälliger lassen sich die übertragenen Wärmen im Kondensator betrachten. Hier scheint es als würde in jeder Messreihe mehr Wärme vom Kältemittel an das Wasser übertragen werden als aus vorangegangenen Berechnungen theoretisch zur Verfügung steht. Auch hier könnten ähnliche Gründe, wie bei der Wärmeübertragung im Verdampfer, zu Erklärungen führen.\\

Die im Kompressor zu betrachtende Wärmübertragung scheint nun wieder einen Sinn zu ergeben, indem die durch das Kältemittel bereitgestellte Wärme nur zu einem Teil an das Kühlwasser abgegeben wird. Die aufgenommene Wärmemenge des Kompressors an das Wasser fällt jedoch lediglich als geringe, oberflächliche Abwärme an. Die Trends der Wärmen sind auch an dieser Stelle im gleichen Maße, wie in den vorangegangen Wärmeübertragungen zu erkennen.\\

Dieser Trend, dass in den Messreihen 1 bis 3 die aufgenommene Wärmemenge an das Wasser steigt, und innerhalb der Messreihen 4 und 5 sinkt zeigt sich auch in der Gesamtwärmeübertragung an das Wasser. Die Summen der Wärmen an das Wasser im Kompressor und im Kondensator bestätigen die berechnete Gesamtwärme. Hierbei fällt auf, dass die Messreihen 3 und 6 den höchsten Betrag  an übertragener Wärme an das Kühlwasser aufweisen.\\

Die Leistungsziffern in Tabelle \ref{tab:auswertung} der Messreihen 1 bis 6 geben als Verhältnis zwischen Nutzwärme und Kompressorleistung an mit welcher Effizienz die Anlage als Wärmepumpe läuft. Hierbei zeigt sich, dass aufgrund der höchsten Leistungsziffer ohne zusätzlich genutzte Kompressionswärme, die Messreihe 2 mit einer Leistungsziffer von 3,9 die höchste Effizienz aufweist. Es folgen danach mit jeweils 3,8 die Messreihe 3 und die Messreihe 6 des luftgekoppelten Verdampfers. Ein eindeutiger Trend der Leistungszahlen innerhalb der Messreihen lässt sich an dieser Stelle nicht erkennen.\\

Als letzten Punkt der Tabelle \ref{tab:auswertung} wird das Druckverhältnis der Messreihen untereinander verglichen. Dieses steigt innerhalb der Messreihen 1 bis 3, ebenso wie in den Messreihen 4 und 5. Messreihe 6 scheint dabei vergleichbar mit Messreihe 3 zu sein. Auffallend ist unter Vergleich dieser Messreihen, dass bis auf Messreihe 3 ähnliche Größenordnungen des Druckverhältnisses auftreten. Ein Grund für diese Abweichung ist aus den gemessenen und berechneten Werten nicht eindeutig erkennbar.

\newpage

In Tabelle \ref{tab:auf4_5_7} sind die verschiedenen Leistungsziffern der Messreihen 1 bis 6 dargestellt. Die Nutzwärme für ersten beiden Zeilen setzt zum einen aus der Kondensationswärme und zum anderen aus der Kondensations- und der Kompressionswärme zusammen. Beides in Bezug auf die Nutzung der Anlage als Wärmepumpe.\\
Hingegen die letzte Leistungsziffer ist auf die Nutzung als Kältemaschine ausgelegt. Hierbei entspricht die Nutzwärme der Zugeführten Wärme in den Verdampfer.\\
Es ist zu erkennen, dass die zusätzliche Kompressionswärme für die Nutzung als Wärmepumpe eine Effizienzsteigerung der Anlage hervorbringt.  Im Vergleich der Leistungsziffern erscheinen hauptsächlich die Messreihen 2 und 6, als die effizientesten Messreihen mit $\varepsilon_{\text{Kond+Komp},2}=4,2$ und $\varepsilon_{\text{Kond+Komp},6}=4,1$. Für die Nutzung als Wärmepumpe sollte demnach die Fahrweise dieser beiden Messreihen unter Nutzung der Kompressions- und der Kondensationswärme bevorzugt werden.\\
Sieht man sich nun die Leistungsziffern in der Nutzung als Kältemaschine an, so ist in den Messreihen 1 bis 5 zu sehen, dass diese eine erhebliche Effizienzeinbuße aufzeigen. Messreihe 6 mit dem luftgekoppelten Verdampfer, statt dem wasserdurchströmten Verdampfer zeigt jedoch ein gegenteiliges Ergebnis. Mit 7,5 ist die Leistungsziffer der Messreihe 6 als Kältemaschine die höchste berechnete Leistungsziffer in diesem Versuch.\\
Daraus lässt sich schließen, dass die Fahrweise der Messreihe 6 sich vorrangig als Kältemaschine eignet und die Messreihen 1 bis 5 als Wärmepumpe. Insgesamt stellt jedoch anhand der Leistungsziffern die Messreihe mit dem luftgekoppelten Verdampfer die beste Performance dar, wenn sowohl der Betrieb als Kältemaschine als auch als Wärmepumpe nötig ist.\\

Es erfolgt die Auswertung des Diagramms \ref{dia:lz_t}, in welchem die Leistungsziffern der Messreihen 1 bis 6 über de jeweiligen Temperaturhub dargestellt sind.\\
Es ist zu erkennen, dass innerhalb der Messreihen 1 bis 2 mit steigendem Temperaturhub auch die Leistungsziffer (nur Kondensation) linear steigt. Ebenso zeigt sich innerhalb dieser Messreihen, dass das Einbeziehen  der Kompressionswärme in die Leistungsziffer diesen linearen Verlauf über den Temperaturhub nicht verändert und scheinbar eine Verschiebung in positive Richtung der $y$-Achse erfolgt. \\
An dieser Stelle fällt auf, dass zu erwarten wäre, dass Messreihe 3 sich ebenfalls in diesen linearen Verlauf der Messreihen 1 und 2 einreiht, jedoch ist dies nicht der Fall. Ein erhöhter Temperaturhub erscheint aus dieser Versuchsdurchführung heraus keinen linearen Zusammenhang mit der Leistungsziffer zu ergeben. Demnach scheint es einen optimalen Temperaturhub zu geben an welchem sich die Leistungsziffer der Messreihe 2 orientieren könnte.  \\
Messreihe 4 und 5 zeigen wiederholt den Gegensatz der Messreihen 1 und 2 auf. Es lässt sich ein linearer Abfall der Leistungsziffern in Abhängigkeit des Temperaturhubes vermuten. Der Einbezug der Kompressionswärme führt hierbei jedoch auch zu einer positiven Verschiebung auf der $y$-Achse.\\
Betrachtet man die Messreihe 6 nun im Vergleich der vorangegangenen Messpunkte im Diagramm, so wird klar, dass diese mit einem vergleichsweise niedrigen Temperaturhub hohe Leistungsziffern als Wärmepumpe einbringt.\\
Auch in dieser Darstellung wird noch einmal stark deutlich wie sehr sich die Leistungsziffern der Messreihen als Kältemaschine voneinander unterscheiden. Alle Messreihen bis auf Messreihe 6 erfahren durch diese Betrachtungsweise der Anlage eine negative Tendenz im Vergleich zur Nutzung als Wärmepumpe. Während sich die Messreihe 1 bis 3 bei einer ähnlichen Leistungsziffer als Kältemaschine dargestellt sind, zeigen sich die Messreihe 4 und 5 mit deutlichen Unterschieden in der Effizienz. Messreihe 5 ist laut dieser Abbildung die schlecht möglichste Fahrweise die Anlage als Kältemaschine zu nutzen.

In Abbildung \ref{dia:tu_t} sind die Überhitzungstemperaturen in Abhängigkeit vom Temperaturhub dargestellt. Im Verlauf der Messreihen 1 bis 3 sinkt diese Überhitzungstemperatur, während sie im Verlauf der Messreihen 4 und 5 steigt. Der Verlauf innerhalb der Messreihen 1 bis 3 wird als nicht linear angenommen und verläuft womöglich asymptotisch. Es fällt auf, dass Messreihe 6 des luftgekoppelten Verdampfers eine deutlich höhere und somit die höchste Überhitzungstemperatur der Messreihen aufweist.\\

In Abb. \ref{dia:q_t} sind die Überhitzungs-, Kondensations- und Heizwärmen der Messreihen 1 bis 6 über den Temperaturhub dargestellt.\\
Beginnend mit den Überhitzungswärmen ist innerhalb der Messreihen 1 bis 3 kein eindeutiger Trend erkennbar. Ein asymptotischer Verlauf könnte maximal vermutet werden. Innerhalb der Messreihen 4 und 5 steigt der Betrag an Überhitzungswärme. Messreihe 6 fällt trotz höher Überhitzungstemperatur im Vergleich zu den restlichen Messreihen als verhältnismäßig gering im Betrag der Überhitzungswärme aus. Die niedrigste Überhitzungswärme hat mit \SI{75}{\watt} Messreihe 2. Aus den vorangegangenen Ausführungen lässt sich die These aufstellen, dass umso geringer die Überhitzungswärme ist, desto effizienter läuft der jeweilige Prozess der Anlage. Somit müsste die Messreihen 2 und 6 die effizientesten Fahrweisen sein, welche sich durch Tabelle \ref{tab:zusatz_m6} bestätigen lässt in Bezug auf die Nutzung als Wärmepumpe.\\
Spiegelbildlich zum Verlauf der Überhitzungswärmen verhält sich der Verlauf der Kondensationswärmen. Es lassen sich umgekehrte Aussagen im Vergleich zu den Überhitzungswärmen formulieren. Demnach sind die Messreihen 2 und 6 für diese Wärme als Maxima beobachtbar.\\
Zuletzt wird der Verlauf der Heizwärmen verglichen. Dieser zeigt für die Messreihen 1 bis 3 einen Verlauf ähnlicher einer steigenden Wurzelfunktion gegenüber dem steigenden Temperaturhub. In den Messreihen 4 und 5 wird jedoch ein Abfall der Heizwärme mit steigendem Temperaturhub beobachtet. Messreihe 6  weißt im Vergleich der Messreihen 1 bis 5 bei niedrigen Temperaturhub eine hohe Heizwärme auf.
Aus dem Diagramm geht hervor, dass die höchsten Heizwärmen von den Messreihe 3 und 6 und die geringsten Heizwärmen für die Messreihen 1 und 5 ausgehen.\\
Schlussendlich lässt sich sagen, dass ausgehend von der Nutzung als Wärmepumpe und unter Betrachtung der Leistungsziffern, Messreihe 2 als Fahrweise der Anlage zu bevorzugen ist. Auch die Überhitzungstemperatur fällt für diese Messreihe, ähnlich wie die Überhitzungswärme als gering aus und lässt sich unter Umständen noch mit Messreihe 6 vergleichen. Messreihe 6 jedoch sollte vorrangig dann als Fahrweise genutzt werden, wenn eine flexible Nutzung als Kältemaschine oder Wärmepumpe zur Debatte steht.