\newpage
\section{Diskussion der Ergebnisse}
\label{sec:diskussion}

In diesem Abschnitt des Protokolls werden die aufgenommenen Messreihen 1 bis 6 untereinander verglichen und ausgewertet.\\

Die in Tabelle \ref{tab:messdaten} aufgeführten Messdaten weisen zunächst keine Auffälligkeiten auf. Lediglich ein Teil aufgenommen Messwerte ist unter Tabelle \ref{tab:mess_luft} zu finden, da Messreihe 6 mit einem luftgekoppelten, statt mit einem wasserdurchströmten Verdampfer aufgenommen wurde.

Beginnend mit dem mittleren Volumen- und Massenstrom fallen diese mit \SI{0,18}{\kmeter \per \second} und \SI{0,22}{\kg \per \per \second} als gering aus. Unter Einbezug der geometrischen Gegebenheiten des Verdampfers und der Luftgeschwindigkeiten erscheinen diese Werte jedoch als plausibel (siehe Tab. \ref{tab:mess_luft}, Gl.  \ref{gl:volumenstrom}, Gl. \ref{gl:massenstrom}).\\

Weiter mit den Ausführungen in Tabelle \ref{tab:auswertung} erscheinen direkt mehrere Punkte von Interesse.\\
Zuerst fällt auf, dass die Kompressorleistung, berechnet mittels Kältemittelmassenstroms und den entsprechenden Kältemittelenthalpie, deutlich geringer ausfällt, als die aus der Messwertaufnahme aufgenommenen Leistungen (vgl. Tab. \ref{tab:messdaten}, Tab. \ref{tab:auswertung}). 
Als Grund hierfür wird vermutet, dass die gemessene Kompressorleistung, der elektrischen, aufzubringenden Leistung  entspricht und somit höher ausfällt. Da diese Leistung auch den damit zusammenhängenden Energieaufwand des Kompressors bemisst, wurden weitere Berechnungen und Betrachtungen auf die höhere, gemessene Kompressorleistung bezogen. Weitere Gründe finden sich unter Abschnitt \ref{sec:fehler}\\
Entsprechend der Erwartung, dass mit steigendem Kältemittelstrom die die aufzubringende aufzubringende Leistung des Verdampfers steigt, bestätigt sich dies innerhalb der Messreihen 1 bis 3. Bei diesen Messreihen ist lediglich der Kühlwasserstrom des Kompressors bzw. Kondensators
bei konstantem Kühlwassermassenstrom durch den wasserdurchströmten Verdampfer verändert worden.\\
In den Messreihen 4 und 5 hingegen wurden der Massenstrom des Kühlwasser für Kompressors bzw. Kondensators konstant gehalten und der Kühlwasserstrom des wasserdurchströmten Verdampfers verändert. Hier ist ebenfalls zu erkennen, dass eine Verringerung des veränderten Kühlwasserstromes eine Verringerung der Kompressorleistung, sowie des Kältemittelmassenstroms hervorruft.\\
Messreihe 6 zeigt hingegen, dass wenn Luft, statt Wasser zum Verdampfen genutzt wird ein deutlich geringerer Massenstrom an Luft genutzt wird. Die gemessene Kompressorleistung, sowie der Kühlwassermassenstrom an Kondensator bzw. Kompressor lassen sich jedoch mit Messreihe 4 vergleichen. \\

Im nächsten Zeilen Abschnitt der Tabelle \ref{tab:auswertung} sind die abgelesenen Kältemittelenthalpien der Punkte 1 bis 4 des $h-\lg{P}$-Diagramms aufgeführt. Sie wurden für weitergehende Wärmeberechnungen genutzt. Angesichts der Größenordnung erscheinen diese Werte als plausibel.\\

Die aufgeführte Verdampfungstemperatur ist ein Messwert, welcher ebenfalls in Tabelle \ref{tab:messdaten} aufgeführt ist. Da R134a ein niedrig verdampfendes Kältemittel ist erscheinen die negativen Temperatur in Grad Celsius als realistisch und können aufgrund der direkten Messwertaufnahme als wahr angenommen werden.\\
Innerhalb der Messreihen 1 bis 3 zeigt sich, dass mit zunehmende Kühlwassermassenstrom innerhalb der Messreihen 1 bis 3 die Verdampfungstemperatur steigt. Innerhalb der Messreihen 4 und 5 sinkt die Verdampfungstemperatur mit steigendem Kühlwasserstrom am Verdampfer. In Messreihe 6 fällt auf, dass diese Messreihe unter Nutzung des luftgekoppelten Verdampfers, die höchste Verdampfungstemperatur des Kältemittels aufweist.\\

Ebenfalls wie die Verdampfungstemperatur steigt auch die aus dem Diagramm bestimmte Kondensationstemperatur der Messreihen 1 bis 3 beim gemessenen Kondensatordruck. Innerhalb der Messreihen 4 und 5 sinkt diese ebenfalls, wie zuvor bei der Verdampfungstemperatur, jedoch im deutlich geringeren Ausmaß. Messreihe 6 zeigt sich mit der selben Kondensationstemperatur, wie Messreihe 5 jedoch mit dem Unterschied, dass die zuvor betrachtete Verdampfungstemperatur in Messreihe 5 knapp \SI{6}{\kelvin} unter der Messreihe 6 des luftgekoppelten Verdampfers liegt.\\

Das spezifische Volumen des Kältemittels am Punkt 1 zeigt sich in den Messreihe 1 bis 3 mit steigendem Kühlwasserstrom an Kondensator bzw. Kompressor als fallend. Auch innerhalb der Messreihen 4 bis 5 die genau umgekehrte Tendenz der Messreihen 1 bis 3 im geringeren Maße zu erkennen. Jedoch zeigt Messreihe 6 einen deutlich erhöhten Wert des spezifischen Volumen an Kältemittel. Dieser ist mit \SI{78}{\kmeter \per \kg} im Vergleich zum höchsten Wert der Messreihen 1 bis 5 mit \SI{30}{\kmeter \per \kg} mer als doppelt so groß.
Der Trend der spezifischen Volumina zeigt infolgedessen auch in ähnlicher Form im Volumenstrom durch den Kompressor.\\

Betrachtet man den Kältemittel Massenstrom so verläuft dieser mit weniger Abweichen geläufig zu den Beobachtungen des spezifischen Volumens aus. Der Kältemittelstrom der Messreihe 6 ist jedoch nicht maßgeblich so auffällig, wie in den vorangegangenen Ausführungen.

Es folgt die Diskussion der übertragenen Wärmen .\\
Innerhalb des Verdampfers wird für die Messreihen 1 bis 5 Wärme vom Wasser und für die Messreihe 6, Wärme der Luft zum verdampfen des Kältemittels übertragen. Zu erwarten ist für die Messreihen 1 bis 3, dass mit steigenden Kühlwasserstrom am Kompressor der Betrag der übertragenen Wärme an das Kältemittel steigt. Für die Messreihen 4 und 5 wird wiederholt ein geläufiger Trend erwartet. Es ist aufgrund von Wärmeverlusten damit zu rechnen, dass die vom Wasser/Luft berechnete, abgegebene Wärme größer ausfällt als die vom Kältemittel aufgenommene Wärme. Diese Vermutung wird durch die Messreihen 1, 3, 4 und 6 bestätigt, aber in den Messreihen 2 und 5 lässt sich eine gegenteilige Meinung begründen. Ein Grund hierfür könnte sich in den Kältemittelenthalpien finden, welche als nicht ausreichend genau abgelesen geltend gemacht werden könnten. Weiterhin ist eine Wärmeaufnahme des Kühlmittels durch die Umgebung selbst nicht ausgeschlossen. Dieser Fehler müsste sich jedoch auf alle Messreihen gleichermaßen auswirken.\\
Noch auffälliger lassen sich die übertragenen Wärmen im Kondensator betrachten. Hier scheint es als würde in jeder Messreihe mehr Wärme vom Kältemittel an das Wasserübertragen werden als aus vorangegangenen Berechnungen theoretisch zur Verfügung steht. Auch hier könnten ähnliche Gründe, wie bei der Wärmeübertragung im Verdampfer zu Erklärungen führen.\\

Die Im Kompressor zu betrachtende Wärmübertragung scheint nun wieder einen Sinn zu ergeben, indem die durch das Kältemittel bereitgestellte Wärme nur zu einem Teil an das Kühlwasser abgegeben wird. Die aufgenommene Wärmemenge des fällt jedoch lediglich als geringe, oberflächliche Abwärme am Kompressor an. Die Trends der Wärmen auch an dieser Stelle im gleich Maße, wie in den vorangegangen Wärmeübertragungen zu erkennen.\\

Dieser Trend, dass in den Messreihen 1 bis 3 die aufgenommene Wärmemenge an das Wasser steigt, und innerhalb der Messreihen 4 und 5 sinkt zeigt sich auch in der Gesamtwärmeübertragung an das Wasser. Die Summen der Wärmen an das Wasser im Kompressor und im Kondensator bestätigen die berechnete Gesamtwärme. Hierbei fällt auf, dass die Messreihen 3 und 6 den höchsten Betrag  an übertragener Wärme an das Kühlwasser aufweisen.\\

Die Leistungsziffern in Tabelle \ref{tab:auswertung} der Messreihen 1 bis 6 geben als Verhältnis zwischen Nutzwärme und Kompressorleistung an mit welcher Effizienz die Anlage als Wärmepumpe läuft. Hierbei zeigt sich, dass aufgrund der höchsten Leistungsziffer ohne zusätzlich genutzte Kompressionswärme, die Messreihe 2 mit 3,9 die höchste Effizienz aufweist. Es folgen danach mit jeweils 3,8 die Messreihe 3 und die Messreihe 6 des luftgekoppelten Verdampfers. Ein eindeutiger Trend der Leistungszahlen innerhalb der Messreihen lässt sich a dieser Stelle nicht erkennen.\\

Als letzten Punkt der Tabelle \ref{tab:auswertung} wird das Druckverhältnis der Messreihen untereinander verglichen. Dieses steigt innerhalb der Messreihen 1 bis 3, ebenso wie in den Messreihen 4 und 5. Messreihe 6 scheint dabei vergleichbar mit Messreihe 3. Auffallend ist unter Vergleich dieser Messreihen, dass bis auf Messreihe 3 ähnliche Größenordnungen des Druckverhältnisses auftreten. Ein Grund für diese Abweichung ist aus den gemessenen und berechneten Werten nicht eindeutig erkennbar.
