\section{Versuchsdurchführung}
\label{sec:durchfuerung}


\subsection*{Durchführung:}
Beginnend mit der Zugabe von \SI{99,3}{\milli \liter} Acetanhydrid und \SI{124,6}{\milli \liter} n-Hexanol in den Kolben war eine sehr leicht gelbliche Lösung zu beobachten. Daraufhin wurden unter Rückkühlung fünf Tropfen konzentrierte Schwefelsäure zugegeben. Hierbei stieg die Temperatur im Kolben von \SI{22}{\celsius} auf bis zu \SI{120}{\celsius} innerhalb von \SI{7}{\minute}. Nach dem erreichen der \SI{120}{\celsius} sank die Temperatur der Lösung im Kolben und es wurde davon ausgegangen, dass die exotherme Reaktion nun am Abklingen war. Es wurde ab diesem Punkt begonnen den Kolben mittels siedendem Wasserbad zu erwärmen. Da sich die ausgewählte Heizplatte in ihrer Heizfunktion als teilweise defekt erwies, wurde diese ausgetauscht und die Temperatur der Lösung sank währenddessen auf \SI{50}{\celsius}. Nach dem die Heizplatte gewechselt war, wurde das zuvor siedende Wasser wieder erwärmt und erneut zum Sieden gebracht. Dieser Vorgang wurde \SI{1}{\hour} lang durchgeführt und ab und zu Wasser im Wasserbad nachgefüllt. Die gemessene Temperatur des Wassers erreichte hierbei maximal \SI{100}{\celsius}. Die Temperatur im Kolben blieb zwischen 95 bis \SI{96}{\celsius}.\\ Zum Ende dieses Versuchsabschnittes ist ausgehend vom Kolbeninhalt ein aromatisch, süßlicher Geruch wahrgenommen worden.
\newpage
\subsection*{Isolierung und Reinigung:}
Nach dem eine Stunde vergangen war, wurde das siedende Wasser des Wasserbades mit Wasser bei Raumtemperatur ausgetauscht, um den Kolben langsam abzukühlen. Der Kolbeninhalt wurde infolgedessen in ein \SI{1}{\liter} Becherglas mit \SI{300}{\milli \liter} Eiswasser gegeben. Darauf hin waren zwei eindeutige Phasen im Becherglas zu erkennen.\\
Der Inhalt des Becherglases ist danach in einen ausreichend großen Scheidetrichter gefüllt worden und die wässrige, untere Phase wurde abgetrennt. Die im Scheidetrichter übergebliebene, organische Phase wurde nun in zwei Teilschritten mit jeweils \SI{40}{\milli \liter} Soda entsäuert. Bei der Zugabe der ersten \SI{40}{\milli \liter} ist eine starke Gasentwicklung zu beobachten, welche am Ende der zweiten Zugabe von \SI{40}{\milli \liter} Soda kaum noch zu erkennen war. Der pH-Wert der organischen Phase wurde am Ausguss des Scheidetrichters mittels Unitest-Papier geprüft und erschien blau. Aufgrund dieser Tatsache wurde beschlossen, dass die organische Phase als entsäuert gilt.\\
Danach wurde die organische Phase mit \SI{50}{\milli \liter} destillierten Wasser gewaschen und anschließend die wässrige Phase erneut mittels Scheidetrichter abgetrennt. Das restliche Wasser in der organische Phase wurde danach mit drei Spatellöffeln Natriumsulfat unter Rühren entfernt und das Natriumsulfat mittels Watte und Trichter abfiltriert.\\
Im Anschluss daran wurde die Apparatur für die Vakuumdestillation aufgebaut. Für die Erwärmung des Kolben Inhaltes wurde an dieser Stelle ein Silikonölbad genutzt. Zudem sind alle Schliffe der Apparatur mit Schlifffett versehen worden, um die Dichtigkeit des Apparates zu gewährleisten. \\
Während der Destillation sind drei Fraktionen aufgefangen worden, welche anschließend mittels einem Refraktometer untersucht und ausgewogen wurden.

\subsection*{Entsorgung}
Das Schlifffett an der Apparatur wurde nach der Destillation mit Essigsäureethylester wieder entfernt. Alles restliche Entsorgungen, wie die wässrige Phase vom Scheidetrichter, das kontaminierte Natriumsulfat, sowie Lösemittel- und Destillationsrückstände sind entsprechend der Versuchsanleitung in den dafür vorgesehenen Abfallbehältern entsorgt worden.