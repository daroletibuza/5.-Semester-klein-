\newpage
\section{Ergebnisse}
\label{sec:ergebnisse}

\begin{flalign}
	A &= \pyr[2]{5*1*2*3+1.1112}{\kilo \meter}
\end{flalign}

\subsection*{Destillationsprotokoll}
Nachfolgend ist das aufgenommene Destillationsprotokoll dargestellt (siehe Tab. \ref{tab:dest_pro}).

% Table generated by Excel2LaTeX from sheet 'Daten'
\begin{table}[h!]
	\renewcommand*{\arraystretch}{1.2}
	\centering
	\rowcolors{2}{gray!25}{white}
	\caption{Destillationsprotokoll}
	\label{tab:dest_pro}
		\begin{tabulary}{1.0\textwidth}{C|C|C|C|C|C}
			\hline
			\textbf{Uhrzeit} & $T_{\text{Öl}} \left[\si{\celsius}\right]$ &$T_{\text{Öl}} \left[\si{\celsius}\right]$ & $p \left[\si{\milli \bar}\right]$&$\frac{\text{Tropfen}}{\text{Sekunde}}$&Fraktion\\
			\hline
			\uhr{13}{57}&23&27&15&0&1 (Vorlauf)\\			
			\uhr{13}{59}&31&27&15&0&1\\
			\uhr{14}{01}&54&27&15&0&1\\
			\uhr{14}{03}&63&27&15&0&1\\
			\uhr{14}{05}&76&28&15&0&1\\
			\uhr{14}{07}&88&28&16&0&1\\
			\uhr{14}{09}&95&28&14&0&1\\
			\uhr{14}{11}&100&63&14&1&1\\
			\uhr{14}{13}&100&64&14&2-3&1\\
			\hline
			\uhr{14}{15}&98&64&14&3&2\\
			\uhr{14}{17}&98&64&14&4&2\\
			\uhr{14}{19}&100&64&14&3&2\\
			\uhr{14}{21}&120&64&14&4&2\\
			\uhr{14}{23}&124&64&14&4&2\\
			\uhr{14}{25}&120&64&14&3-4&2\\
			\uhr{14}{27}&119&64&14&3&2\\
			\hline
			\uhr{14}{29}&115&64&14&2-3&3\\
			\uhr{14}{31}&105&64&14&1&3\\
			\uhr{14}{33}&100&62&14&0&3\\
			\hline
	\end{tabulary}
\end{table}%
\FloatBarrier

\subsection*{Brechungsindices} 
Um die drei entstandenen Fraktionen in der Zusammensetzung miteinander vergleichen zu können wurden diese mittels Refraktometer untersucht. Dabei wurden der Brechungsindex der jeweiligen Fraktion bestimmt, welche unter Tabelle \ref{tab:brechung} dargestellt sind.

% Table generated by Excel2LaTeX from sheet 'Daten'
\begin{table}[h!]
	\renewcommand*{\arraystretch}{1.2}
	\centering
	\rowcolors{2}{white}{gray!25}
	\caption{Brechungsindices der Fraktionen 1 bis 3}
	\label{tab:brechung}
		\begin{tabulary}{1.0\textwidth}{C|CCC}
			\hline
			\textbf{Fraktion} & \textbf{1} & \textbf{2}&\textbf{3}\\
			\hline
			\textbf{Brechungsindex}&1,410&1,409&1,409\\
			\hline			
	\end{tabulary}
\end{table}%
\FloatBarrier 

\newpage

\subsection*{Ausbeute}
Die zu Beginn des Versuches abgemessenen \SI{99,3}{\milli \liter} Acetanhydrid und \SI{124,6}{\milli\liter} n-Hexanol entsprachen Stoffmengen von \SI{1,05}{\mol} und \SI{1}{\mol}. Die Masse an entstandenen Ester wurde mit \SI{112,98}{\gram} eingewogen und eine molare Masse von \SI{144,21}{\gram \per \mole} für den Ester bestimmt. Aus diesen Angaben wurden die Ausbeute des Essigsäure-n-hexylesters in Gleichung \ref{gl:ausbeute} bestimmt. 
\begin{flalign}
	\label{gl:ausbeute}
	\eta 	&= \frac{n_{\text{Ester}}}{n_{\text{n-Hexanol}}} = \frac{m_{\text{Ester}}}{n_{\text{n-Hexanol}}*M_{\text{Ester}}}\\[2mm]
	&=	\frac{\SI{112,98}{\gram}}{\SI{1,00}{\mol}*\SI{144,21}{\gram \per \mole}}\\
	&=\underline{\SI{78,3}{\percent}}
\end{flalign}



