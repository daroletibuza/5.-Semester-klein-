\newpage
\section{Diskussion der Ergebnisse}
\label{sec:diskussion}

Beginnend mit den Versuchen zur Löslichkeit des Rohproduktes ließ sich feststellen, dass sich das Produkt nicht in Wasser, zum Teil in Ethanol und vollständig in Essigsäureethylester löste (siehe Tab. \ref{tab:löslichkeit}). Grund hierfür werden die unterschiedlichen Polaritäten der Lösemittel sein. Da das Produkt selbst, durch seine aromatische Struktur (siehe Abb. \ref{fig:mechanismus}) eher unpolar ist, neigt es dazu sich auch eher in unpolaren Lösemitteln zu lösen. Dies bestätigt sich mit der Tatsache, dass Essigsäureethylester das unpolarste der getesteten Lösemittel ist und sich dort eine gute Löslichkeit verzeichnen lässt. Ebenso ist die umgekehrte Begründung wirksam, dass im vergleichsweise, polarsten Lösungsmittel Wasser sich das unpolare Produkt am schlechtesten lösen lässt.

%\anmerkung{Lösemittelabhängigkeit für die Addition}
%?cyclohexan: aprotisches Lösemittel --> keine Abgabe von H+ --> keine Reaktion mit Br-
%?wasser/Ethanol: protisch LM --> Abgabe von H+ --> Reaktion zu HBr

Die Schmelzpunkte zwischen Fraktion 1 und 2 unterscheiden sich eindeutig \mbox{(siehe Tab. \ref{tab:schmelzpunkte})}. Die Tatsache, dass Fraktion 2 einen höheren Schmelzbereich besitzt, lässt darauf schließen, dass in dieser Fraktion Anteilig mehr Reinprodukt vorhanden ist, als in Fraktion 1. Begründen lässt sich dies dadurch, dass organische Verbindungen mit einem höheren Anteil an Halogenen generell höhere Schmelzpunkte besitzen. Diese Tatsache beruht auf den elektrostatischen Wechselwirkungen der Moleküle zum Beispiel in Form von Dipol-Dipol-Wechselwirkungen. Zudem wirken die höheren molaren Massen der Halogenatome auch auf eine höhere molare Masse der Moleküle und verstärken somit auch die \textsc{Van-der-Waals}-Kräfte zwischen den Molekülen. Je mehr dieser elektrostatischen Kräfte zwischen den Molekülen wirken, desto mehr Energie ist für die Überwindung dieser Kräfte nötig, um Phasenübergang zu ermöglichen. Diese benötigte Energie wird im höheren Schmelzpunkt deutlich.\\

An dieser Stelle wird das auftreten von Nebenprodukten in diesem Versuch diskutiert. Diese gelten als Verunreinigungen für das Hauptprodukt (1,2-Dibrom-1-phenylethan). In dieser Diskussion wird angenommen, dass in diesem Versuch neben der gewünschten Additionsreaktion ebenfalls eine Substitutionsreaktion auftritt. Diese kann auftreten, wenn die Reaktion bei hohen Temperaturen abläuft oder das Brom radikalisch durch Absorption von Photonenenergie zerfällt (Stichwort: radikalische Substitution). In Abb. \ref{fig:nebenprodukte} sind mögliche Nebenprodukte einer solchen Substitution dargestellt. \\
\bild{Reaktion zu Nebenprodukten einer radikalische Substitution (vereinfacht)}{nebenprodukte}{0.6}
\newpage

Unterscheiden lassen sich Haupt- und Nebenprodukt durch die unterschiedlichen Schmelzpunkte der Verbindungen. Diese Unterscheiden sich wesentlich \mbox{(siehe Tab. \ref{tab:schmelzpunkte})}. Somit können die Fraktionen allein durch die Schmelzpunkte einfach unterschieden werden.\\
Es gibt jedoch auch noch die Möglichkeit massenspektroskopisch die Verbindungen zu unterscheiden. Da die Nebenprodukte und das Hauptprodukt sich in einem Bromatom unterscheiden, würde sich in einer MS-Analyse ein Unterschied in den Molaren Massen der Proben ergeben. Das Hauptprodukt hätte hierbei die höhere molare Masse. Die Annahme, dass in Fraktion 2 zu 100\% Reinprodukt vorhanden ist, könnte auf diese Weise ebenfalls überprüft werden.\\

Die mögliche Ausbeute von \SI{65}{\percent} Reinprodukt bzw. der allgemeine Umsatz von \SI{84}{\percent} wird vermutlich hauptsächlich durch die Bildung von Nebenprodukten und Rückständen in den jeweiligen Messgeräten und Apparaturen zu erklären sein. Diese Erklärungen geben auch einen Hinweis darauf, dass die Annahme von 100\% Reinstoff in Fraktion 2 in der Realität vermutlich nicht zutreffen wird.

