\newpage
\section{Diskussion der Ergebnisse}
\label{sec:diskussion}

Beginnend mit den Versuchen zur Löslichkeit des Produktes ließ sich feststellen, dass sich das Produkt nicht in Wasser, zum Teil in Ethanol und vollständig in Essigsäureethylester löste. Grund hierfür werden die unterschiedlichen Polaritäten der Lösemittel sein. Da das Produkt selbst, durch seine aromatische Struktur (siehe Abb. \ref{fig:mechanismus}) eher unpolar ist, neigt es dazu sich auch eher in unpolaren Lösemitteln zu lösen. Die bestätigt sich mit der Tatsache, dass Essigsäureethylester das unpolarste Lösemittel ist und sich dort eine gute Löslichkeit verzeichnen lässt. Ebenso ist die umgekehrte Begründung wirksam, dass im vergleichsweise, polarsten Lösungsmittel sich das unpolare Produkt am schlechtesten lösen lässt.

\anmerkung{Lösemittelabhängigkeit für die Addition}
?cyclohexan: aprotisches Lösemittel --> keine Abgabe von H+ --> keine Reaktion mit Br-
?wasser/Ethanol: protisch LM --> Abgabe von H+ --> Reaktion zu HBr

\anmerkung{Schmelzpunktdiskussion}\\
Höher bei mehr Halogenen, da mehr Dipol Dipol Wechselwirkungen und höhere Van der Waal WW höhere Masse durch Halogene

\anmerkung{Diskussion von verschiedenen Produkten}\\
An dieser Stelle wird das auftreten von Nebenprodukten in diesem Versuch diskutiert. Diese gelten als Verunreinigungen für das Hauptprodukt (1,2-Dibrom-1-phenylethan). In dieser Diskussion wird angenommen, dass in diesem Versuch neben der gewünschten Additionsreaktion ebenfalls eine Substitutionsreaktion auftritt. Diese kann auftreten, wenn die Reaktion bei hohen Temperaturen abläuft oder das Brom radikalisch durch Absorption von Photonenenergie zerfällt (Stichwort: radikalische Substitution). In Abb. \ref{fig:nebenprodukte} sind mögliche Nebenprodukte einer solchen Substitution dargestellt. \\
\bild{Reaktion zu Nebenprodukten einer radikalische Substitution (vereinfacht)}{nebenprodukte.png}{0.9}
Unterscheiden lassen sich Haupt- und Nebenprodukt durch die unterschiedlichen Schmelzpunkte der Verbindungen. Diese Unterscheiden sich stark siehe \anmerkung{Verweis !}. Somit könnte die Fraktionen allein durch die Schmelzpunkte einfach unterschieden werden.\\
Es gibt jedoch auch noch die Möglichkeit massenspektroskopisch die Verbindungen zu unterscheiden. Da die Nebenprodukte und das Hauptprodukt sich in einem Bromatom unterscheiden, würde sich in einer MS-Analyse ein Unterschied in den Molaren Massen der Proben ergeben. Das Hauptprodukt hätte hierbei die höhere molare Masse.

Die Ausbeute von \anmerkung{\SI{0,0}{\percent}} wird vermutlich hauptsächlich durch Rückstände in den jeweiligen Messgeräten und Apparaturen zu erklären sein. \anmerkung{unvollständiger Stoffumsatz ??}\\ 

