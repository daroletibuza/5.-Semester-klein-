\newpage
\section{Ergebnisse}
\label{sec:ergebnisse}

\subsection*{Schmelzpunkt}
\begin{table}[h!]
	\renewcommand*{\arraystretch}{1.2}
	\centering
	\rowcolors{2}{white}{gray!25}
	\caption{Schmelzpunkte der Produktfraktionen}
	\label{tab:schmelzpunkte}
	%\resizebox{10.5cm}{!}{
	\begin{tabulary}{1.0\textwidth}{C|CC}
		\hline
		\textbf{Fraktion} & \textbf{Fraktion 1} & \textbf{Fraktion 2}\\
		\textbf{Schmelzpunkt in $\left[\si{\celsius}\right]$} &&\\
		\hline			
	\end{tabulary}
	%}
\end{table}%
\FloatBarrier

\subsection*{Löslichkeitsprüfung}

\begin{table}[h!]
	\renewcommand*{\arraystretch}{1.2}
	\centering
	\rowcolors{2}{white}{gray!25}
	\caption{Löslichkeit des Rohproduktes in Wasser, Ethanol und Essigsäureethylester}
	\label{tab:löslichkeit}
	%\resizebox{10.5cm}{!}{
	\begin{tabulary}{1.0\textwidth}{C|C|C|C}
		\hline
		\textbf{Lösemittel} & \textbf{Wasser} & \textbf{Ethanol} & \textbf{Essigsäureethylester} \\
		\hline
		\textbf{Rohprodukt} & nicht löslich & schwer löslich & löslich\\
		\hline			
	\end{tabulary}
	%}
\end{table}%
\FloatBarrier

\subsection*{Ausbeute}
Die zu Beginn des Versuches abgemessenen \SI{11,5}{\milli \liter} frisch destilliertes Styrol und \SI{5,0}{\milli\liter} Brom entsprachen Stoffmengen von jeweils \SI{0,1}{\mol}. Die Massen der jeweiligen Produktfraktionen wurden eingewogen und sind in Tabelle \ref{tab:massen} aufgeführt. Molare Massen möglicher Nebenprodukte (siehe Abb. \ref{fig:nebenprodukte}) und des Hauptproduktes (1,2-Dibrom-1-phenylethan) sind mit \SI{182}{\gram \per \mole} und \SI{262}{\gram \per \mol} bestimmt worden.

\begin{table}[h!]
	\renewcommand*{\arraystretch}{1.2}
	\centering
	\rowcolors{2}{white}{gray!25}
	\caption{Massen der Produktfraktionen für die Bromierung von Styrol}
	\label{tab:massen}
	%\resizebox{10.5cm}{!}{
		\begin{tabulary}{1.0\textwidth}{C|CC|C}
			\hline
			\textbf{Fraktion} & \textbf{Masse} & \textbf{Stoffmenge} & \textbf{Notiz}\\
			\hline
			Ansatz& - &\SI{0,1}{\mol}&-\\
			Produktfraktion 1 & & &\\
			Produktfraktion 2 & & &\\
			Rohprodukt & & & Summe aus Fraktion 1+2\\
			Produkt & & & aus \SI{1}{\gram} Rohprodukt\\
			\hline
			theoretisches Produkt &&&\makecell{theoretisch, mögliches,\\ reines Produkt}\\
			\hline			
	\end{tabulary}
	%}
\end{table}%
\FloatBarrier

Aus diesen Angaben ergeben sich die folgenden Berechnungen zur Ausbeuteberechnung: 

\subsection*{Berechnung der Masse des Rohproduktes}
\textit{$\rightarrow$ analoge Berechnung für die Stoffmenge}
\begin{flalign}
	m_{RP}	&= m_{F1}+m_{F2}\\
	&= \\
	&= \underline{}
\end{flalign}

\subsection*{Berechnung des Umsatzgrades in Rohprodukt}
\begin{flalign}
	\eta_{F1}	&= \frac{n_{F1}}{n_{\ce{Br2}}}=\\
	\eta_{F2}	&= \frac{n_{F2}}{n_{\ce{Br2}}} =\\
	\eta_{RP}	&= \frac{n_{F1}+n_{F2}}{n_{\ce{Br2}}} =\\
\end{flalign}


\subsection*{Berechnung des Verunreinigungsgrades}
\begin{flalign}
	\eta_V	&= \frac{m_P(\SI{1}{\gram}) }{m_{RP}(\SI{1}{\gram})}\\
	&= \\
	&= \underline{}
\end{flalign}

\subsection*{Berechnung des theoretischen Hauptproduktes im Rohprodukt}
\begin{flalign}
	n_P	&= \frac{\eta_V*m_{RP}}{M_{P}}\\
	&= \\
	&= \underline{}
\end{flalign}

\subsection*{Berechnung des theoretisch möglichen Umsatzgrades für das Hauptprodukt nach einer Umkristallisation}
\begin{flalign}
	\eta &= \frac{n_P}{n_{\ce{Br2}}}\\
		&=
\end{flalign}


%\anmerkung{Tabelle: Was wurde alles ausgewogen}
%Start: 0,1 mol
%Fraktion1: Reaktion 1 (Molare Masse HP oder NP benutzen)
%Fraktion2: Reaktion 2
%Rohprodukt(Fraktion 1+2): Ausbeute was generell umgesetzt wurde
%Produkt (aus 1g Rohprodukt): Ausbeute was durch Umkristallisation übrig bleibt --> Hochrechnung auf Rohprodukt


