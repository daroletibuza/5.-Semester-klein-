\section{Einleitung und Versuchsziel}
\label{sec:aufgabenstellung}
%In der Aufgabenstellung wird (in eigenen Worten und ganzen Sätzen) formuliert, was das Ziel des 
%Versuches ist.  
%[Beachten Sie die eigentliche Aufgabenstellung in den Versuchsanleitungen sowie die Hinweise zur Auswertung!] 

Im folgenden Versuch wird aus Styrol und Brom in Anwesenheit von Cyclohexan 1,2-Dibrom-1-phenylethan dargestellt. Hauptsächlich wird in diesem Versuch die arbeitsmethodische Kenntnis zum Umkristallisieren benötigt . Das entsprechende Produkt wird mittels Schmelzpunkt untersucht und so die einzelnen Fraktionen miteinander verglichen. Weiterhin erfolgt in diesem Protokoll die Diskussion, wie sich weitere denkbare Reaktionsprodukte mittels spektroskopischer Daten und einfachen Versuchen ausschließen lassen.\\
Nachfolgend ist der Mechanismus, der Reaktion zu 1,2-Dibrom-1-phenylethan dargestellt.

\bild{Mechanismus der Addition von Brom an Styrol}{mechanismus}{1.0}