\newpage
\section{Versuchsdurchführung}
\label{sec:durchfuerung}
\subsection*{Durchführung:}
Zu Beginn wurde in einem \SI{250}{\milli \liter} mit Magnetrührer \SI{11,5}{\milli \liter} frisch destilliertes Styrol in \SI{100}{\milli \liter} Cyclohexan vorgelegt. Die Mischung wurde \SI{10}{\minute} unter Rühren in einem 8-\SI{10}{\celsius} kalten Wasserbad gekühlt. Danach tropft man eine Mischung aus \SI{50}{\milli \liter} Cyclohexan und \SI{5}{\milli \liter} Brom dazu. Durch die Reaktion des Broms an die Doppelbindung des Styrols ist ein Verschwinden des rotbraunen Bromdampfes zu beobachten. Die Lösung verfärbt sich dabei mit der Zeit hellbraun. Zuletzt wurde die Mischung weitere \SI{15}{\minute} beim Raumtemperatur gerührt und ein Feststoff fällt als Produkt aus.

\subsection*{Isolierung und Reinigung:}
Das ausgefallene Produkt wurde mit einem \textsc{Büchner}-Trichter abgesaugt. Der Rückstand wird ausgepresst und stellt die erste Produktfraktion dar. Für die zweite Produktfraktion wurde aus dem Filtrat das Lösemittel mit Hilfe eines Rotationsverdampfers bei verminderten Druck abdestilliert.\\
Danach wurden beide Fraktionen auf einem Tonteller unter einem Abzug für eine Stunde getrocknet. Nach der Trocknung wurden die Fraktionen ausgewogen und die Schmelzpunkte bestimmt. Beide Fraktionen sind danach zu einem Rohprodukt zusammengeführt worden.\\
Es folgte die Prüfung der Löslichkeit des Rohprodukts. Hierfür werden je \SI{500}{\milli \gram} des Rohproduktes in \SI{1}{\milli \liter} Lösungsmittel (Wasser, Ethanol, Essigsäureethylester) gelöst.\\

Zur Aufreinigung des Rohproduktes schloss sich der Löslichkeitsprüfung eine Umkristallisation an. Hierfür wurde zunächst ein Gemisch von Ethanol und Wasser im Verhältnis 7:3 hergestellt. Für die eigentliche Umkristallisation wurde \SI{1}{\gram} des Rohproduktes und \SI{7}{\milli \liter} des Ethanol-Wasser-Gemisches in einen \SI{50}{\milli \liter} Kolben gegeben. Der Inhalt des Kolbens wurde unter Rückfluss erhitzt bis sich der Kolbeninhalt gelöst hatte. Die Lösung wurde danach zuerst auf Raumtemperatur und dann im 8-\SI{10}{\celsius} kalten Wasser abgekühlt. Das auskristallisierte, gereinigte Produkt ist dann mit einem \textsc{Hirsch}-Trichter abgesaugt worden. Zum Schluss wurde mit kalten Ethanol-Wasser-Gemisch nachgespült. Das Produkt ist danach ausgewogen worden.

\subsection*{Entsorgung}
Alle mit Brom verunreinigten Geräte wurden unter dem Abzug liegen gelassen, um das restliche Brom abdampfen zu lassen. Die wasserfreie Mutterlaugen, welche bei der Umkristallisation und den Löslichkeitsprüfungen entstanden, wurden im Behälter für halogenhaltige Abfälle entsorgt. Das Lösemittel, welches mittels Rotationsverdampfer getrennt wurde, ist dem Recycling zugeführt worden.
