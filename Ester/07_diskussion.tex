\newpage
\section{Diskussion der Ergebnisse}
\label{sec:diskussion}

Eine Ausbeute von \SI{15,48}{\percent} erscheint realistisch, da Nelken die ätherischen Öle lediglich als flüchtige Komponenten und nicht Strukturbausteine der Blüten enthalten.\\

Die Ergebnisse aus der Gaschromatografie (siehe Tab. \ref{tab:zusammen}) ergeben, dass sich hauptsächlich drei Verbindungen im Nelkenöl befinden. Dies deckt sich mit den einleitenden Worten des Protokolls unter Abschnitt \ref{sec:aufgabenstellung}. 
Anhand des vergleichsweise großen Anteils von \SI{76,2}{\percent} im Nelkenöl lässt sich vermuten, dass Verbindung 1 aus Tabelle \ref{tab:zusammen} die Verbindung Eugenol ist. Vergleicht man die gemessene molare Masse der ersten Verbindung mit der des Eugenols, so sind diese mit \SI{164}{\gram \per \mole} identisch \cite{Berger.2017}. Das unterstützt die These, dass Verbindung 1 mit dem höchsten Peak im Chromatogramm Eugenol entspricht.\\
Die zweite Verbindung mit \SI{204}{\gram \per \mole} könnte somit dem $\beta$-Caryophyllen entsprechen \cite{ROMPPRedaktion.2002}. Der Anteil von \SI{11,5}{\percent} deckt sich ebenfalls mit der Angabe der Literatur (ca. 10\% und mehr) und bestätigt somit, dass Verbindung 2 dem $\beta$-Caryophyllen entspricht \cite{Krammer.2003}.\\
Die dritte Verbindung weist mit \SI{206}{\gram \per \mol} die molare Masse von Eugenolacetat auf. In der Literatur mit 5-\SI{10}{\percent} angegeben, scheint der Anteil von \SI{12,3}{\percent} als plausibel \cite{Krammer.2003}. Weitere Quellen sprechen auch von bis zu \SI{17}{\percent} Anteil an Eugenolacetat \cite{Wikipedia.2020}.

Somit ergibt sich, dass das extrahierte Nelkenöl wie zu Beginn erwartet die Verbindungen Eugenol, Eugenolacetat und $\beta$-Caryophyllen enthält. Die überarbeitete Tabelle \ref{tab:zusammen} ist nun unter Tabelle \ref{tab:zusammen_neu} wiederzufinden.

\begin{table}[h!]
	\renewcommand*{\arraystretch}{1.2}
	\centering
	\rowcolors{2}{white}{gray!25}
	\caption{Gaschromatische und massenspektroskopische Daten des Nelkenöls}
	\label{tab:zusammen_neu}
	\resizebox{10.5cm}{!}{
		\begin{tabulary}{1.0\textwidth}{C|C|C|C}
			\hline
			\textbf{Verbindung} & \textbf{Retentionszeit} $\left[\si{\minute}\right]$& \textbf{Molare Masse} $\left[\si{\gram \per \mole}\right]$&\textbf{Anteil} $\left[\si{\percent}\right]$\\
			\hline
			Eugenol & 26,43 & 164 & 76,2\\
			$\beta$-Caryophyllen & 28,15 & 204 & 11,5\\
			Eugenolacetat & 30,24 & 206 & 12,3\\
			\hline      
	\end{tabulary}}
\end{table}%
\FloatBarrier