\section{Ergebnisse}
\label{sec:ergebnisse}

\subsection*{Ausbeute}

Mit den \SI{60}{\milli \mol} Brombenzol und der Probenmasse von 4-Nitrobrombenzol mit \SI{4,22}{\gram} berechnet sich in Gleichung \ref{gl:ausbeute_4} der Gehalt an ausgefallenen 4-Nitrobrombenzol für diese Versuchsdurchführung. Das über Rotationsverdampfen gewonnene Gemisch aus 2- und 4-Nitrobrombenzol besitzt eine Masse aus \SI{3,39}{\gram}.
\begin{flalign}
	\label{gl:ausbeute_4}
	\eta_{\text{4-NBB}}	&= \frac{m_{\text{4-NBB, Probe}}}{M_{\text{4-NBB}}*n_{\text{Brombenzol}}}= \frac{n_{\text{4-NBB, Probe}}}{n_{\text{Brombenzol}}}\\[2mm]
				&= \frac{\SI{4,22}{\gram}}{\SI{0,16205}{\gram \per \milli \mol}*\SI{60,00}{\milli \mol}} = \frac{\SI{26,04}{\mmol}}{\SI{60,00}{\milli \mol}}\\
				&=	\underline{\SI{43,4}{\percent}}
\end{flalign}

\subsection*{Dünnschichtchromatografie} 
Mittels Dünnschichtchromatografie werden unter diesem Abschnitt das Rohprodukt mit Reinstprodukten, sowie der Einfluss der Polarität des Fließmittels auf die DC verglichen.
Nach Gleichung \ref{gl:rf} ergeben sich für die $R_f$-Werte der Proben:
\begin{flalign}
	\label{gl:rf}
	R_f  &= \frac{\text{Laufstrecke der Substanz}}{\text{Laufstrecke des Lösungsmittels}}
\end{flalign}
\begin{flalign}
	R_f (\text{2-NBB, rein, 8:1}) &= \frac{\SI{4,2}{\centi\meter}}{\SI{6,5}{\centi \meter}}=\underline{0,65}
\end{flalign}

Zusammengefasst sind die Ergebnisse der DC in Tabelle \ref{tab: dc} dargestellt.

\begin{table}[h!]
	\renewcommand*{\arraystretch}{1.2}
	\centering
	\rowcolors{2}{gray!25}{white}
	\caption{$R_f$-Werte für das Rohprodukt, reinem 2- und 4-Nitrobrombenzol}
	\label{tab: dc}
	\resizebox{15cm}{!}{
		\begin{tabulary}{1.0\textwidth}{C|C|C|C|C}
			\hline
			& \textbf{Punkt 1 (roh)} & \textbf{4-NBB (rein)} &\textbf{Punkt 2 (roh)} & \textbf{2-NBB (rein)}\\
			\hline
			8:1 Strecke LM &\multicolumn{4}{c}{\SI{6,5}{\centi\meter}}\\
			14:1 Strecke LM &\multicolumn{4}{c}{\SI{6,6}{\centi\meter}}\\
			\hline
			8:1 Strecke Substanz & \SI{5,4}{\centi \meter} &  \SI{5,6}{\centi \meter} & \SI{4,2}{\centi \meter} & \SI{4,2}{\centi \meter} \\
			14:1 Strecke Substanz & \SI{5,2}{\centi \meter} &  \SI{5,3}{\centi \meter} & \SI{3,1}{\centi \meter} & \SI{3,2}{\centi \meter} \\
			\hline
			8:1 $R_f$-Wert & 0,83 &  0,86& 0,65 &0,65\\
			14:1 $R_f$-Wert & 0,79 &  0,80& 0,47 &0,48\\
			\hline      
	\end{tabulary}
}
\end{table}%
\FloatBarrier


\newpage
\subsection*{Schmelzpunktbestimmung}
Die Ergebnisse der Schmelzpunktbestimmung für das ausgefallene 4-Nitrobrombenzol und dem über Rotationsverdampfen gewonnenen Gemisch aus 2- und 4-Nitrobrombenzol sind in Tabelle \ref{tab:schmelz} aufgeführt.

\begin{table}[h!]
	\renewcommand*{\arraystretch}{1.2}
	\centering
%	\rowcolors{2}{white}{gray!25}
	\caption{Schmelzbereiche der Proben}
	\label{tab:schmelz}
	\resizebox{15cm}{!}{
		\begin{tabulary}{1.0\textwidth}{C|C|C}
			\hline
			&\textbf{4-NBB (Probe)} & \textbf{2- bzw. 4-NBB-Gemisch (Probe)}\\
			\hline
			Schmelzbereich &117-\SI{119}{\celsius}&34-\SI{35}{\celsius}\\
			\hline      
	\end{tabulary}}
\end{table}%
\FloatBarrier




