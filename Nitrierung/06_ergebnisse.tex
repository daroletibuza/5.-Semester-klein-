\section{Ergebnisse}
\label{sec:ergebnisse}

\subsection*{Ausbeute}

Mit den \SI{60}{\milli \mol} Brombenzol und der Probenmasse von 4-Nitrobrombenzol mit \SI{4,22}{\gram} berechnet sich in Gleichung \ref{gl:ausbeute_4} der Gehalt an ausgefallenen 4-Nitrobrombenzol für diese Versuchsdurchführung.
\begin{flalign}
	\label{gl:ausbeute_4}
	\eta_{\text{4-NBB}}	&= \frac{m_{\text{4-NBB, Probe}}}{M_{\text{4-NBB}}*n_{\text{Brombenzol}}}= \frac{n_{\text{4-NBB, Probe}}}{n_{\text{Brombenzol}}}\\[2mm]
				&= \frac{\SI{4,22}{\gram}}{\SI{0,16205}{\gram \per \milli \mol}*\SI{60,00}{\milli \mol}} = \frac{\SI{26,04}{\mmol}}{\SI{60,00}{\milli \mol}}\\
				&=	\underline{\SI{43,4}{\percent}}
\end{flalign}

\subsection*{Dünnschichtchromatografie} 



\subsection*{Schmelzpunktbestimmung}


\begin{table}[h!]
	\renewcommand*{\arraystretch}{1.2}
	\centering
	\rowcolors{2}{white}{gray!25}
	\caption{Zusammengefasste Ergebnisse der Gaschromatografie und der \mbox{Massenspektroskopie}}
	\label{tab:zusammen}
	\resizebox{10.5cm}{!}{
		\begin{tabulary}{1.0\textwidth}{C|C|C|C}
			\hline
			\textbf{Verbindung} & \textbf{Retentionszeit} $\left[\si{\minute}\right]$& \textbf{Molare Masse} $\left[\si{\gram \per \mole}\right]$&\textbf{Anteil} $\left[\si{\percent}\right]$\\
			\hline
			Verbindung 1 & 26,43 & 164 & 76,2\\
			Verbindung 2 & 28,15 & 204 & 11,5\\
			Verbindung 3 & 30,24 & 206 & 12,3\\
			\hline      
	\end{tabulary}}
\end{table}%
\FloatBarrier




