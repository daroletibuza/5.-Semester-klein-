\section{Versuchsdurchführung}
\label{sec:durchfuerung}

\begin{figure}[h!]
	\centering
	\includegraphics[width=0.8\textwidth]{img/versuchsaufbau}
	\caption{Schematischer Versuchsaufbau der Nitrierung}
	\label{fig:versuchsaufbau}
\end{figure}
\FloatBarrier
%Ende
\vspace*{-5mm}

\subsection*{Durchführung:}
Der Versuch begann mit der Einwaage von \SI{25,03}{\gram} handelsüblichen Nelkenblüten. Diese wurden mittels elektrischer Laborschlagmühle fein zu einem Pulver zerkleinert.
Daraufhin wurde das Nelkenblütenpulver zusammen mit \SI{200 }{\milli \liter} destilliertes Wasser in einen \SI{1}{\liter}-Zweihalskolben gegeben.
Es erfolgte der restliche Aufbau der Destillationsapparatur mit \textsc{ Liebig}-Kühler, Thermometer, einem zu $\frac{2}{3}$ mit vorgewärmten destillierten Wasser gefüllten \SI{2}{\liter}-Rundkolben und einem Auffangkolben für das Destillat.
Zusätzlich wurden dem \SI{2}{\liter}-Kolben vier Siedesteinchen zugegeben.\\
Das Thermometer zeigte zu Beginn des Versuches eine Temperatur von \SI{22}{\celsius}.\\
Nun wurden der \SI{2}{\liter}-Kolben mit dem Wasser und der \SI{1}{\liter}-Kolben mit der Nelkenpulversuspension jeweils mit einem Heizpilz erwärmt.
Sobald das Wasser im \SI{2}{\liter}-Einhalskolben siedete, wurden beide Rundkolben über einen Schlauchstück-Aufsatz miteinander verbunden, sodass der Wasserdampf in den Kolben der Nelkensuspension gelangte.\\
Nach fünf Minuten ist am Thermometer eine Temperatur des austretenden Dampfes von \SI{98}{\celsius} abzulesen und die Suspension mischte sich stark mit dem Wasserdampf.
Im \textsc{Liebig}-Kühler kondensierte der Dampf aus und der Rundkolben füllte sich langsam mit einer weiß-trüben Emulsion.\\

Im weiteren Verlauf der Wasserdampfdestillation wies die Temperatur einen Wert von \SI{100}{\celsius} des austretenden Dampfes auf.
Die Temperatur wurde daraufhin nun ca. alle zehn Minuten überprüft und blieb konstant bei den genannten  \SI{100}{\celsius}.
Nach dem der erste Auffangkolben fast gefüllt war, wurde die erste Fraktion im Eisbad gekühlt und ein weiterer Auffangkolben montiert.
Es ließ sich beobachten, dass sich die 2. Fraktion ebenfalls als weiß-trüb beschreiben lässt. Eine Stunde nach Beginn der Wasserdampfdestillation befand sich im Destillat keine Trübung mehr. Die zweite Fraktion wurde nun ebenfalls gekühlt, welche optisch gleich der 1. Fraktion erschien.

\subsection*{Isolierung und Reinigung:}
Nach der durchgeführten Wasserdampfdestillation wurde die Apparatur abgebaut und der Inhalt der beiden Auffangkolben in den großen Scheidetrichter zusammen mit \SI{20}{\milli \liter} Cyclohexan gegeben.\\
Nach dem ersten Schütteln im Scheidetrichter bildete sich eine feine Emulsion aus Cyclohexan, den ätherischen Ölen und Wasser.
Um die ätherischen Öle im Cyclohexan Zwang zu lösen, wurden diese mittels Zugabe von \SI{100}{\milli \liter} gesättigter Natriumchlorid-Lösung aus dem Wasser verdrängt und der Scheidetrichter nun erneut geschüttelt.
Im Scheidetrichter war nun oberhalb eine ölige Phase und unterhalb die restliche Wasser-Öl-Emulsion zu beobachten. Es erfolgte das Abscheiden der Cyclohexan-Öl- Mischung.\\

Die Extraktion wird auf diese Weise, ohne weitere Zugabe von Natriumchlorid-Lösung weitere zwei Male wiederholt, bis lediglich eine erkennbare Emulsionsphase übrig blieb.
Das Extrakt aus Cyclohexan und ätherischen Öl wurde nun mit acht gehäuften Spatelspitzen Natriumsulfat versetzt, um das Wasser aus dem Cyclohexan zu binden. Die trübe Cyclohexan-Öl-Mischung wurde dadurch klar. 

Danach wurde das abgesetzte Natriumsulfat mittels Watte und Glastrichter abfiltriert. Es folgte die Destillation im Rotationsverdampfer bei \SI{70}{\celsius} und \SI{265}{\milli \bar} Absolutdruck.

In dieser Zeit wird ein weiterer, kleinere Kolben ergänzend mit \SI{33,796}{\gram} leer eingewogen.
Die vordestillierte Lösung wurde in den kleineren Kolben gegeben und erneut bei \SI{70}{\celsius} und \SI{265}{\milli \bar} Absolutdruck wiederholt destilliert.
Nach zehn Minuten erfolgte eine Anpassung der Temperatur und  des Druckes auf \SI{60}{\celsius} und \SI{200}{\milli \bar}.
Nach weiteren zehn Minuten wurden der Auffangbehälter für das Cyclohexan in den organischen Abfällen entsorgt und die Destillation erneut begonnen. Die eingestellte Temperatur betrug \SI{60}{\celsius} und der Druck \SI{80}{\milli \bar} absolut.\\

Nachdem die Destillation durch die Laborbetreuung als abgeschlossen galt, wurde die Lösung in den vorgewogenen Rundkolben gegeben und es ließen sich \SI{37,676}{\gram} an Masse messen.
In der Differenz zum Leergewicht ergab sich somit eine Probenmenge von \SI{3,87}{\gram} ätherischen Öls.\\
Ein Teil des Öls wurde anschließend entnommen und für die Gaschromatografie vorbereitet.

\subsection*{Entsorgung:}
Die Nelkensuspension wird in einem Kunststofftrichter mit Filterpapier abfiltriert. Der Filterkuchen wurde im Hausmüll und das Filtrat im Abfluss entsorgt.\\
Das Cyclohexan wurde im organischen Abfallbehälter entsorgt.