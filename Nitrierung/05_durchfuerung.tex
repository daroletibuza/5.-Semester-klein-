\section{Versuchsdurchführung}
\label{sec:durchfuerung}

\begin{figure}[h!]
	\centering
	\includegraphics[width=0.8\textwidth]{img/versuchsaufbau}
	\caption{Schematischer Versuchsaufbau der Nitrierung}
	\label{fig:versuchsaufbau}
\end{figure}
\FloatBarrier
%Ende
\vspace*{-5mm}

\subsection*{Durchführung:}
Zu Beginn des Versuches wurden \SI{6}{\milli \liter} konzentrierte Schwefelsäure im Rundkolben vorgelegt und mittels Eisbad auf \SI{10}{\celsius} gekühlt. Es folgte die Zugabe der konzentrierten Salpetersäure. Da das Küken des Feststofftrichters zuvor nicht ausreichend auf Funktionalität und benötigtes Schlifffett geprüft wurde, erfolgte zu Beginn keine tropfenweise Zugabe der Salpetersäure. Die Temperatur stieg somit kurzzeitig auf \SI{20}{\celsius} an. Nachdem der Inhalt des Kolbens wieder auf \SI{10}{\celsius} sank, wurde nun tropfenweise weitere Salpetersäure zugegeben. Die Lösung innerhalb des Kolbens färbte sich mit der Zeit gelblich. \\
Die Nitriersäure im Kolben wurde daraufhin auf \SI{15}{\celsius} erwärmt und es wurde mit der Zugabe von Brombenzol in die Lösung begonnen. Diese erfolgte in ca. \SI{2,3}{\minute}-Intervallen à \SI{0,5}{\milli \liter} Brombenzol-Lösung. Ab einer zugegebenen Menge von \SI{2}{\milli \liter} Brombenzol ließ sich im Kolben eine gelblich, weiße Ausflockung beobachten. Nachdem insgesamt \SI{6,5}{\milli \liter} Brombenzol zugegeben wurden, waren vorwiegend um den Rührfisch im Kolben die zuvor beschriebenen Ausflockungen zu erkennen. Während der Zugabe das Brombenzols blieb die Temperatur innerhalb des Kolbens zwischen 26-\SI{35}{\celsius}.\\
Die Lösung wurde daraufhin auf \SI{55}{\celsius}  mittels Wasserbad erwärmt und \SI{45}{\minute} lang auf dieser Temperatur gehalten. Es erfolgte danach ein Abkühlen des Kolbeninhaltes auf Raumtemperatur. Der Kolbeninhalt wurde daraufhin in ein Becherglas mit \SI{60}{\milli \liter} Eiswasser gegeben und Reste innerhalb des Kolbens mit einem Spatel ausgekratzt.\\
\newpage
Anschließend wurde der Inhalt des Becherglases mit einem \textsc{Büchner}-Trichter abgesaugt und es blieb ein gelber Filterkuchen und ein farbloses Filtrat über. Infolgedessen wurde der Filterkuchen in Wasser gegeben und erneut abgesaugt. Das wurde ein weiteres Mal wiederholt und zum Schluss noch einmal abgesaugt. Nach dem dritten Mal absaugen wurde der Filterkuchen auf einem Tonteller getrocknet und anschließend erneut in einen Rundkolben gegeben und mit ca. \SI{30}{\milli \liter}  \SI{95}{\percent}-igen Ethanol umkristallisiert. Der Heizpilz wurde hierbei auf Stufe 3 gestellt.\\
Das aus dem Ethanol ausgefallene 4-Nitrobrombenzol wurde danach heiß in ein Becherglas gegeben und infolgedessen erneut abgesaugt und mit kaltem Ethanol gewaschen. Das Ethanol wurde dafür kurzzeitig in ein Eisbad gestellt. Das Filtrat dieses Absaugens wurde danach in einem Rotationsverdampfer abdestilliert.

\subsection*{Isolierung und Reinigung}
Während dem Umkristallisieren des Rohproduktes wurden ein Teil des Rohproduktes aus 2- und 4- Nitrobrombenzol, sowie Reinstproben von 2-Nitrobrombenzol und 4-Nitrobrombenzol mit \SI{2,7}{\milli \gram}, \SI{4,0}{\milli \gram} und \SI{4,2}{\milli \gram} in jeweils drei Reagenzgläser eingewogen. Jedem Reagenzglas wurde danach jeweils \SI{1}{\milli \liter} Essigsäureethylester zugegeben.\\
Nachdem die Proben für die Dünnschichtchromatografie vorbereitet waren, wurden zwei Fließmittel à \SI{10}{\milli \liter} hergestellt. Diese enthielten ein Gemisch aus Cyclohexan und Essigsäureethylester. Im ersten Fließmittel wurde ein Verhältnis von 8:1 im zweiten ein Verhältnis von 14:1 gewählt. Mittels Kapillaren wurden nun zwei DC-Platten jeweils mit Rohprodukt, 2-Nitrobrombenzol und 4-Nitrobrombenzol betüpfelt. Die Entwicklung der DC-Platten erfolgte in zwei separaten Kammern mit den jeweiligen Fließmitteln. \\
Zum Schluss wurden die entwickelten DC-Platten unter einer UV-Lampe untersucht.\\
Zu einem separaten Termin wurden die Massen der Produkte, sowie deren Schmelzpunkte bestimmt.

\subsection*{Entsorgung:}
Das wässrige Filtrat vom Rohprodukt, sowie die wässrige Phase beim Waschen wurde im Sammelbehälter für halogenhaltige Abfälle entsorgt. Die Überreste der Reagenzgläser mit den Laufmitteln wurden ebenfalls im halogenhaltigen Abfall entsorgt. Ansonsten wurden jegliche Glasapparaturen bzw. Glasgeräte mit Aceton gespült.

\newpage