\newpage
\section{Diskussion der Ergebnisse}
\label{sec:diskussion}

Die Ausbeute mit \SI{43,4}{\percent} an 4-Nitrobrombenzol erscheint sinnvoll, da sie knapp über dem Erwartungswert von 35-\SI{40}{\percent} der Versuchsanleitung liegt. Mögliche Verunreinigungen durch beispielsweise nicht umgesetztem Edukt könnten diese Überschreitung des Erwartungswertes erklären.\\

Bei der Dünnschichtchromatografie zeigt sich im Fall beider Fließmittel, dass das Rohprodukt aus 2- und 4-Nitrobrombenzol besteht. Zu erkennen ist dies daran, dass das Rohprodukt sich in jeweils zwei Punkte aufteilt und diese ähnliche $R_f$-Werte besitzen, wie die Reinstproben. Die jeweils längeren Strecken der Substanzen entsprechen dem 4-Nitrobrombenzol.\\
Trotz ähnlicher Ergebnisse zeigen die beiden unterschiedlichen Fließmittel auch unterschiedliche $R_f$-Werte auf. Das polarer Fließmittel mit einem Verhältnis von 8:1 an Cyclohexan und Essigsäureethylester zeigt dabei die größeren $R_f$-Werte auf. Somit ist eine höhere Beweglichkeit der Moleküle im Fließmittel zu verzeichnen. So ergibt es sich, dass im Fließmittel mit der 14:1 Mischung an Cyclohexan und Essigsäureethylester eine bessere Trennleistung ergibt, da die Abstände zwischen dem den 4-Nitrobrombenzol- und den 2-Nitrobrombenzol-Punkten größer sind. Grund hierfür könnte sein, dass das 2-Nitrobrombenzol im unpolaren Lösungsmittel aufgrund seiner räumlichen Struktur größeren Wechselwirkungen ausgesetzt ist, als das 4-Nitrobenzol, welches eine vergleichsweise lineare Struktur aufweist.\\
Es wird demnach die 14:1 Mischung als Fließmittel empfohlen. \\

Aus den Schmelzbereichen der Tab. \ref{tab:schmelz} zeigt sich im Vergleich mit Literaturwerten, dass die Produkte der Nitrierung nicht vollkommen rein sind (siehe Tab. \ref{tab:schmelzvergleich}). Als Literaturwert für das 2- bzw. 4-NBB-Gemisch wird reines 2-Nitrobrombenzol genutzt. Zu erkennen sind vorliegenden Verunreinigungen an der Tatsache, dass die Verunreinigung von Reinstoffen meist eine Schmelzpunkterniedrigung mit sich führt \cite{ROMPPRedaktion.2002b}. Im Fall vom 2-Nitrobrombenzol (im Gemisch) gilt zumindest das 4-Nitrobrombenzol als Verunreinigung. \\
Eine weitere Bestimmung der Verunreinigungen der Proben erfordert weitere analytische Untersuchungen.

\begin{table}[h!]
	\renewcommand*{\arraystretch}{1.2}
	\centering
	%	\rowcolors{2}{white}{gray!25}
	\caption{Schmelzbereiche der Proben}
	\label{tab:schmelzvergleich}
	\resizebox{15cm}{!}{
		\begin{tabulary}{1.0\textwidth}{C|C|C}
			\hline
			Schmelzbereich&\textbf{4-NBB (Probe)} & \textbf{2- bzw. 4-NBB-Gemisch (Probe)}\\
			\hline
			Messwert  &117-\SI{119}{\celsius}&34-\SI{35}{\celsius}\\
			Literatur \cite{Wikipedia.2019} &124-\SI{126}{\celsius}&40-\SI{42}{\celsius} (2-NBB)\\
			\hline      
	\end{tabulary}}
\end{table}%
\FloatBarrier
