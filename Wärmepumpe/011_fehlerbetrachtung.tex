\section{Fehlerbetrachtung}
\label{sec:fehler}

Als Fehlerquellen in der Versuchsdurchführung sind hauptsächlich die elektrischen Sensoren festzustellen, welche durch Eigenwiderstände oder äußere, ungewollte Umwelteinflüsse eine Verfälschung der Messwerte verursachen können. Die aufgenommenen Messwerte erscheinen jedoch realistisch und werden demnach auch als valide eingestuft. Größeres Vertrauen wird in die analoge Messung der Massenströme und der Druck- sowie Temperaturmessung der Umgebung gelegt. Diese hatten keine weiteren, eindeutigen Einflussfaktoren auf die Messung im Vergleich zur elektrischen Messwertaufnahme. Jedoch könnte die Bewegung der Schwebekörper durch den Durchfluss minimale Ablesefehler verursacht haben. \\

Weiterhin ist bei der Anlage damit zu rechnen, dass der Kühlwasserstrom nicht konstant war. Da dieser dem Wasserleitungssystem des Labors entspringt, sind bei der Nutzung des Wassers in anderen Räumen auch Schwankungen zu erwarten. Diese sind jedoch nicht direkt in den Messwerten oder in der Auswertung dieser Versuchsdurchführung erkennbar gewesen.\\
Zu dem ist in der Anlage auch damit zu rechnen, dass Verlustwärmen auftreten, welche an die Umgebung abgegeben werden. Es ist jedoch ebenfalls möglich, dass Wärme der Umgebung entzogen wurde und diese zu Abweichungen in den erwarteten Bilanzergebnissen führte.\\

Zuletzt sind in der Auswertung Fehler aufgrund von Ableseungenauigkeiten durch das $h-\lg{P}$-Diagramm nicht auszuschließen. Hier wird die größte Fehlerquelle der Versuchsauswertung vermutet und ist somit der erste Punkt der bei einer erneuten Auswertung der Messdaten überprüft und gegebenenfalls überarbeitet werden sollte. 

