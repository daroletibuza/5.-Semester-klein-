\section{Diskussion der Ergebnisse}
\label{sec:diskussion}

Eine Ausbeute von \SI{95}{\percent} erscheint realistisch, da mit jedem Arbeitsschritt Rückstande in und an den Arbeitsgeräten zurückbleiben bzw. haften. Ebenfalls zu beachten sind die Messtoleranzen an den Messzylinder und Waagen, welche jedoch keine große Fehlerquelle darstellen dürften.\\

Anhand des Vergleiches der $R_f$-Werte zwischen Referenzsubstanz und Produkt lässt sich grob die Reinheit des Produktes bewerten. Da die $R_f$-Werte von 2-Naphthol und Orange II mit 0,918 und 0,411 sich relativ unterschiedlich sind, ist davon auszugehen, dass sich das 2-Naphthol in der Synthese zum Großteil umgesetzt hat. Da jedoch das 2-Naphthol keinen eindeutigen Endpunkt für die Dünnschichtchromatografie aufwies, sollte die genutzte 4:1 Essigsäureethylester-Methanol-Lösung als Laufmittel hinterfragt werden. Für genauere Untersuchungen ist unter Umständen ein alternatives Laufmittel zu verwenden.
