\section{Versuchsdurchführung}
\label{sec:durchfuerung}
\begin{figure}[h!]
	\centering
	\includegraphics[width=0.8\textwidth]{img/versuchsaufbau_1}
	\caption{Allgemeiner Aufbau der Versuchsapparatur}
	\label{fig:versuchsaufbau_1}
\end{figure}
\FloatBarrier
%Ende

\subsection{Versuchsdurchlauf  1: }
Beginnend mit der Einwaage von \SI{1,71}{\gram} Sulfanilsäure, in ein \SI{100}{\milli \liter} Becherglas, führte eine Zugabe von \SI{18}{\milli \liter} \SI{2,5}{\percent}iger Natriumcarbonat-Lösung zum Aufschäumen im Becherglas.
Dabei löste sich die Sulfanilsäure teilweise. Um die Sulfanilsäure vollständig zu lösen wurde ein Ultraschallbad genutzt, um das Salz weiter zu zerkleinern und somit die Löslichkeit zu unterstützen.\\
Infolgedessen sollte die Lösung in einem Eisbad, unter Rühren mittels Magnetrührer, auf \SI{0}{\celsius} gekühlt werden. Da sich mit dem vorhanden Eis die gewünschte Temperatur nicht erreichen ließ, wurde weiteres Eis zugegeben. Da dabei Eis nicht nur in der Eisschale, sondern auch in die Lösung gelangte, musste der Versuch erneut angesetzt werden.

\subsection{Versuchsdurchlauf 2: }
\subsubsection*{Durchführung 2:}
Der erneute Versuchsdurchlauf beginnt mit der Einwaage von \SI{1,70}{\gram} Sulfanilsäure in ein neues \SI{100}{\milli \liter} Becherglas. Es folgt, wie im vorangegangen Durchlauf, die Zugabe von \SI{18}{\milli \liter} \SI{2,5}{\percent}iger Natriumcarbonat-Lösung unter Aufschäumen der Lösung. Wiederholt löst sich die Sulfanilsäure nicht vollständig und es erfolgt ein Ultraschallbad zur Verbesserung der Löslichkeit. Nach dem Lösen der Sulfanilsäure soll diese nun ebenfalls wieder auf \SI{0}{\celsius} gekühlt werden und wird unter Rühren in ein Eisbad gegeben. Nach \SI{10}{\min} fällt die Temperatur von \SI{20}{\celsius} auf \SI{3}{\celsius}. Da sich an dieser Stelle die Temperatur kaum noch zu ändern scheint, wird weiteres Eis zugegeben. Zusätzlich wird Wasser zur Vergrößerung der Kontaktfläche dem Eisbad hinzufügt.\\
Da auf diese Weise die Temperatur der Lösung auf \SI{0}{\celsius} nicht erreichbar erscheint, wird mithilfe von Natriumchlorid und dem Eisbad, eine Kältemischung hergestellt. Nach weiteren \SI{3}{\min} Wartezeit sind nun die \SI{0}{\celsius} der Lösung erreicht worden.\\
Unter Rühren wurden nun \SI{0,80}{\gram} Natriumnitrit der Lösung hinzugegeben, welche sich darauf gelblich färbt. Während die gelbe Lösung nun weitere \SI{5}{\min} rührt, wurde  in ein \SI{400}{\milli \liter} Becherglas \SI{9}{\gram} Eis mit \SI{2}{\milli \liter} 1N-Salzsäure gegeben. Das \SI{400}{\milli \liter} Becherglas wurde darauf hin, statt der gelben Lösung im Eisbad gerührt und gekühlt. Danach wurde die gelbe Lösung portionsweise zugegeben. Die Temperatur wurde dabei unter \SI{8}{\celsius} gehalten. Die Lösung im Becherglas erscheint nun als dunkel-gelb. Es fiel hierbei kein Salz aus, entgegen der Beschreibung in der Versuchsanleitung.\\
Zusätzliche drei Tropfen an konzentrierter Salzsäure trübten die Lösung und es wurde mit der Versuchsdurchführung fortgefahren. \\

\subsubsection*{Isolierung und Reinigung 2:}
Nach der Einwaage von \SI{1,4}{\gram } rosa 2-Naphthol in \SI{8}{\milli \liter} \SI{10}{\percent}ige farbloser Natronlauge entsteht eine braune Lösung. Das 2-Naphthol hat sich an dieser Stelle teilweise gelöst und so wurde das \SI{100}{\milli \liter} Becherglas zum vollständigen Lösen in ein Ultraschallbad gestellt.\\
Die braune Lösung wurde nun mittels Eisbad auf \SI{3}{\celsius} gekühlt und danach in die Diazoniumsalzlösung gegeben. Zudem wurde aufgrund einer Temperatur von \SI{6}{\celsius} in der Lösung weiteres Eis in die Eisschale gegeben. Die Lösung färbt sich rot-orange bis dunkelrot. Dabei fällt wider der Erwartung kein Farbstoff aus. Auch eine Zugabe von festem Methylorange als Impfkristall, nach Absprache mit den Versuchsbetreuern,   verschaffte kein ausfällen des Farbstoffes.\\
An dieser Stelle wurde sich entschlossen den Versuch noch einmal zu wiederholen.

\subsection{Versuchsdurchlauf 3: }
\subsubsection{Durchführung 3:}
Wie in den vorangegangen Durchläufen auch, werden zunächst \SI{1,70}{\gram} Sulfanilsäure eingewogen. Die Sulfanilsäure, die diesmal benutzt wurde, ist jedoch frisch zur Verfügung gestellt worden. Nach der Zugabe von \SI{18}{\milli \liter} \SI{2,5}{\percent}iger Natriumcarbonat-Lösung beginnt die Lösung erneut zu schäumen und die die Sulfanilsäure löst sich nicht vollständig. Unter Hilfenahme des Ultraschallbades löst sich die Sulfanilsäure nun vollständig.\\
Die Lösung wird nun direkt in einer Eis-Natriumchlorid-Kältemischung auf \SI{0}{\celsius} gekühlt. Es erfolgt die Einwaage von \SI{0,81}{\gram} Natriumnitrit, welches unter Rühren der Lösung portionsweise zugegeben wird. Die Lösung färbt sich gelb und trübt sich.\\
Nun wurden \SI{8,97}{\gram} Eis eingewogen und zusammen mit \SI{2}{\milli \liter} 1N-Salzsäure gemischt. Die Eis-Säure-Lösung wird nun unter Rühren gekühlt und die gelbe, trübe Lösung hinzugegeben. Die Lösung färbt sich gelb-orange. Es fällt kein Salz aus.\\
Auf Anraten der Versuchsbetreuer wurde erneut \SI{1}{\milli \liter} 1N-Salzsäure zugegeben und die Lösung färbte sich etwas dunkler. Es wurde die Vermutung nahe gelegt, dass der pH-Wert zu hoch sei und daraufhin weitere \SI{2}{\milli \liter} konzentrierte Salzsäure bei \SI{5}{\celsius} zugegeben. Der Verdacht bestätigt sich und es fällt ein gelbes Niederschlag bzw. ein Salz aus. Es stellte sich heraus, dass in der Praktikumsanleitung selbst auch von konzentrierter Salzsäure die Rede ist, jedoch die angegebene Übersicht der Chemikalien, sowie die bereitgestellte Salzsäure im Praktikum lediglich 1N war.\\
Nach dieser Erkenntnis und dem erfolgreichen Ausfällen des Diazoniumsalzes folgte nun die weitere Versuchsdurchführung.
\subsubsection*{Isolierung und Reinigung 3:}
Im Schritt der Isolierung und Reinigung wurden im dritten Versuchsdurchlauf \SI{1,29}{\gram} 2-Naphthol mit \SI{8}{\milli \liter} \SI{10}{\percent}iger Natronlauge vermengt und es bildete sich eine braune Lösung, in welcher das 2-Naphthol nicht vollständig gelöst wurde.