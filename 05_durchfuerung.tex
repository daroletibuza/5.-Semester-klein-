\section{Versuchsdurchführung}
\label{sec:durchfuerung}
\begin{figure}[h!]
	\centering
	\includegraphics[width=0.6\textwidth]{img/versuchsaufbau_1}
	\caption{Allgemeiner Aufbau der Versuchsapparatur}
	\label{fig:versuchsaufbau_1}
\end{figure}
\FloatBarrier
%Ende

\subsection{Versuchsdurchlauf  1: }
Beginnend mit der Einwaage von \SI{1,71}{\gram} Sulfanilsäure, in ein \SI{100}{\milli \liter} Becherglas, führte eine Zugabe von \SI{18}{\milli \liter} \SI{2,5}{\percent}iger Natriumcarbonat-Lösung zum Aufschäumen im Becherglas.
Dabei löste sich die Sulfanilsäure teilweise. Um die Sulfanilsäure vollständig zu lösen wurde ein Ultraschallbad genutzt, um das Salz weiter zu zerkleinern und somit den Lösevorgang zu beschleunigen.\\
Infolgedessen sollte die Lösung in einem Eisbad, unter Rühren mittels Magnetrührer, auf \SI{0}{\celsius} gekühlt werden. Da sich mit dem vorhanden Eis die vorgegebene Temperatur nicht erreichen ließ, wurde weiteres Eis zugegeben. Da dabei Eis nicht nur in der Eisschale, sondern auch in die Lösung gelangte, musste der Versuch erneut angesetzt werden.

\subsection{Versuchsdurchlauf 2: }
\subsubsection*{Durchführung 2:}
Der erneute Versuchsdurchlauf beganns mit der Einwaage von \SI{1,70}{\gram} Sulfanilsäure in ein neues \SI{100}{\milli \liter} Becherglas. Es folgte, wie im vorangegangen Durchlauf, die Zugabe von \SI{18}{\milli \liter} \SI{2,5}{\percent}iger Natriumcarbonat-Lösung unter Aufschäumen der Lösung. Wiederholt löste sich die Sulfanilsäure nicht vollständig und es erfolgt ein Ultraschallbad zur Beschleunigung des Lösevorganges. Nach dem Lösen der Sulfanilsäure soll diese nun ebenfalls wieder auf \SI{0}{\celsius} gekühlt werden und wurde unter Rühren in ein Eisbad gegeben. Nach \SI{10}{\min} fiel die Temperatur von \SI{20}{\celsius} auf \SI{3}{\celsius}. Da die Temperatur nicht auf die vorgegebenen \SI{3}{\celsius} sank, wurde weiteres Eis zugegeben. Zusätzlich ist Wasser zur Vergrößerung der Kontaktfläche dem Eisbad hinzufügt worden.\\
\newpage
Da auf diese Weise die Temperatur der Lösung auf \SI{0}{\celsius} nicht erreichbar erscheint, wurde mithilfe von Natriumchlorid und dem Eisbad eine Kältemischung hergestellt. Nach weiteren \SI{3}{\min} Wartezeit sind nun die \SI{0}{\celsius} der Lösung erreicht worden.\\
Unter Rühren wurden nun \SI{0,80}{\gram} Natriumnitrit der Lösung hinzugegeben, welche sich darauf gelblich färbte. Während die gelbe Lösung nun weitere \SI{5}{\min} gerührt wurde, ist  in ein \SI{400}{\milli \liter} Becherglas \SI{9}{\gram} Eis mit \SI{2}{\milli \liter} 1N-Salzsäure gegeben worden. Das \SI{400}{\milli \liter} Becherglas wurde darauf hin, statt der gelben Lösung im Eisbad gerührt und gekühlt. Danach wurde die gelbe Lösung portionsweise zugegeben. Die Temperatur wurde dabei unter \SI{8}{\celsius} gehalten. Die Lösung im Becherglas erschien nun als dunkel-gelb. Entgegen der Beschreibung in der Versuchsanleitung fiel hierbei kein Salz aus.\\
Zusätzliche drei Tropfen an konzentrierter Salzsäure trübten die Lösung und die Versuchsdurchführung wurde fortgesetzt. \\

\subsubsection*{Isolierung und Reinigung 2:}
Nach der Einwaage von \SI{1,4}{\gram } rosafarbenen 2-Naphthol in \SI{8}{\milli \liter} \SI{10}{\percent}ige farbloser Natronlauge entsteht eine braune Lösung. Das 2-Naphthol hat sich teilweise gelöst und so wurde das \SI{100}{\milli \liter} Becherglas zum vollständigen Lösen in ein Ultraschallbad gestellt.\\
Die Diazoniumsalzlösung wurde nun mittels Eisbad auf \SI{3}{\celsius} gekühlt und danach unter Rühren in die braune 2-Naphthol-Lösung gegeben. Zudem wurde aufgrund einer Temperatur von \SI{6}{\celsius} in der Lösung weiteres Eis in die Eisschale gegeben. Die Lösung färbt sich rot-orange bis dunkelrot. Dabei fiel wider der Erwartung kein Farbstoff aus. Auch eine Zugabe von festem Methylorange als Impfkristall, nach Absprache mit den Versuchsbetreuern,   verschaffte kein Ausfällen des Farbstoffes.\\
An dieser Stelle wurde beschlossen,der den Versuch noch einmal zu wiederholen.

\subsection{Versuchsdurchlauf 3: }
\subsubsection*{Durchführung 3:}
Wie in den vorangegangen Durchläufen auch, werden zunächst \SI{1,70}{\gram} Sulfanilsäure eingewogen. Die Sulfanilsäure, die diesmal benutzt wurde, ist jedoch frisch zur Verfügung gestellt worden. Nach der Zugabe von \SI{18}{\milli \liter} \SI{2,5}{\percent}iger Natriumcarbonat-Lösung begann die Lösung erneut zu schäumen und die Sulfanilsäure löste sich nicht vollständig, bis zur Nutzung des Ultraschallbades.\\
Die Lösung wurde nun direkt in einer Eis-Natriumchlorid-Kältemischung auf \SI{0}{\celsius} gekühlt. Es erfolgte die Einwaage von \SI{0,81}{\gram} Natriumnitrit, welches unter Rühren der Lösung portionsweise zugegeben wurde. Die Lösung färbte sich gelb und trübte sich.\\
Nun wurden \SI{8,97}{\gram} Eis eingewogen und zusammen mit \SI{2}{\milli \liter} 1N-Salzsäure gemischt. Die Eis-Säure-Lösung wurde nun unter Rühren gekühlt und die gelbe, trübe Lösung hinzugegeben. Die Lösung färbte sich gelb-orange. Es fällte kein Salz aus.\\
\newpage
Auf Anraten der Versuchsbetreuer wurde erneut \SI{1}{\milli \liter} 1N-Salzsäure zugegeben und die Lösung färbte sich etwas dunkler. Es wurde die Vermutung nahe gelegt, dass der pH-Wert zu hoch sei und daraufhin weitere \SI{2}{\milli \liter} konzentrierte Salzsäure bei \SI{5}{\celsius} zugegeben. Der Verdacht bestätigte sich und es fällte ein gelber Niederschlag bzw. ein Salz aus.\\
Ursache für die vorangegangenen Fehlversuche war eine Diskrepanz in der Versuchsvorschrift. Es unterschieden sich somit die bereitgestellten mit den zu benutzenden Chemikalien. Dass lediglich 1N Salzsäure, statt konzentrierter Salzsäure zur Verfügung gestellt wurde, ist vom Protokollführer weder erkannt, noch angesprochen worden.\\
Nach dieser Erkenntnis und dem erfolgreichen Ausfällen des Diazoniumsalzes folgte nun die weitere Versuchsdurchführung.

\subsubsection*{Isolierung und Reinigung 3:}
Im Schritt der Isolierung und Reinigung wurden im dritten Versuchsdurchlauf \SI{1,29}{\gram} 2-Naphthol mit \SI{8}{\milli \liter} \SI{10}{\percent}iger Natronlauge vermengt und es bildete sich eine braune Lösung, in welcher das 2-Naphthol nicht vollständig gelöst wurde. Erneut kam das Ultraschallbad zum Einsatz. Die Diazoniumsalzaufschlämmung wurde nun auf \SI{3}{\celsius} gekühlt und wurde unter Rühren der 2-Naphthol-Lösung zugegeben. Es entstand eine orange-rote, dickflüssige Schlämmung.\\
Diese Aufschlämmung wurde daraufhin bei \SI{45}{\celsius} erhitzt, wurde flüssiger und die Farbe änderte sich in ein stark dunkles Rot. Nachdem sich der Farbstoff nach \SI{5}{\min } vollständig gelöst hatte, wurden bei \SI{40}{\celsius} \SI{5,04}{\gram} Natriumchlorid zugegeben, welches sich ebenfalls löste.\\ 
Danach wurde das Gemisch auf \SI{20}{\celsius} an der Umgebung abgekühlt und daraufhin weitere \SI{30}{\min} in einem Eisbad auf \SI{3}{\celsius} herunter gekühlt. Nach etwa \SI{10}{\min} wurde der abgekühlte Inhalt des Becherglases dem Aufbau der Vakuumfiltration zugeführt und mit wenig gesättigter Natriumchlorid-Lösung gewaschen. Ein fester, oranger Feststoff, blieb als Filterkuchen übrig.\\
Über einen Rundkolben, einen Rückflusskühler und \SI{10}{\milli \liter} 3:1 Wasser-Ethanol- Gemisch wurde der Feststoff für \SI{10}{\min} umkristallisiert und anschließend heiß in ein Becherglas gegeben. Der Farbstoff kühlte sich ab und kristallisiert wieder aus.\\
Die darauffolgende Vakuumfiltration wurde im Nachgang dieses Versuches durch die Laborbetreuung durchgeführt, da zum Ende des Praktikums die Vakuumpumpen außer Betrieb waren.
\newpage
\subsubsection*{Charakterisierung 3:}
Nach 5 Tagen wurde ein leeres Glas ohne Deckel mit \SI{12,26}{\gram} und danach zusammen mit dem filtrierten Präparat \SI{15,23}{\gram} eingewogen. In der Subtraktion ergibt sich somit eine Probenmasse von \SI{2,97}{\gram}. \\
Für die Charakterisierung des Farbstoffes im Vergleich zum Ausgangsstoff wurde daraufhin eine Untersuchung mittels Dünnschichtchromatografie durchgeführt. Dafür wurden \SI{5}{\milli \gram} 2-Naphthol und \SI{5}{\milli \gram} des hergestellten Orange II in jeweils \SI{1}{\milli \liter} Methanol gelöst. Als Laufmittel wurden \SI{10}{\milli \liter } einer 4:1 Essigsäureethylester-Methanol-Lösung genutzt. Nach dem Betropfen mit den zwei Lösungen wurde die DC-Platte mit Kieselgel (\textsc{Merck} 60) für \SI{3}{\min} in den leeren Schacht der Entwicklungskammer mit dem Laufmittel gelagert. Nach dieser Zeit wurde die Platte in den Schacht mit dem Laufmittel gegeben, bis sich eine Laufmittelfront zum Papierrand von ca. \SI{1}{\centi \meter} gebildet hatte. Das Fließverhalten der Stoffe ließ sich am Farbstoff erkennen.\\
Nun konnte unter UV-Licht bei \SI{254}{\nano\meter} der Start- bzw. der Endpunkt der jeweiligen Lösung für die Berechnung des $R_f$-Wertes ermittelt werden. 

\subsubsection*{Entsorgung 3:}
Die Entsorgung erfolgte in den entsprechenden Abfallbehältern für halogenhaltige und nicht-halogenhaltige, organische Abfälle, sowie über die Ausgüsse an den Arbeitsplätzen für ungefährliche Stoffe.