\section{Durchführung}
\label{sec:durchfuerung}

Um den Einfluss der Rührerdrehzahl auf den Absorptionsprozess zu untersuchen, wird zunächst das Absorptionsmittel, in diesem Fall Leitungswasser, in den Rührbehälter gegeben.
Das Thermostat, welches eine konstante Wassertemperatur gewährleistet, wird auf \SI{20}{\celsius} eingestellt. Danach folgen das Anbringen des Rührers, des gasführenden Metallröhrchens, sowie der Sauerstoffsonde, welche für dieses Praktikum bereits in den entsprechenden Halterungen am Rührkessel montiert wurden. 
Bei langsamer Drehzahl wird nun überprüft ob der Rührer gegen die zuvor montierte Sauerstoffsonde oder das Metallröhrchen schlägt.

Nach dem Aufbauen der Apparatur wird nun gewartet bis sich das Wasser mit drehenden Rührer auf \SI{20}{\celsius} erwärmt. Überprüft wird dies mit der Sauerstoffsonde, welche ebenfalls einen Temperaturfühler besitzt.
Gleichzeitig wird mittels Stickstoff der sich im Leistungswasser befindende Sauerstoff ausgetrieben. Dies erfolgte im Praktikum nicht vollständig, sondern jeweils bis zu einem Wert von rund \SI{0,18}{\milli \gram \per \liter} an Sauerstoff.

Nun kann mit dem eigentlichen Versuch begonnen werden und es wird bei festgelegter Drehzahl der Luftsauerstoff konstant zugeführt. Gemessen wird der Volumenstrom an zugeführter Luft mittels Schwebekörperdurchflussmesser in \si{\liter \per \min}. Im Praktikum wurde sich auf einen Volumenstrom von rund \SI{0,50}{\liter \per \minute} festgelegt.
Mit Beginn der Luftzufuhr wird ebenfalls die Messung bzw. das Logging der Temperatur und der Sauerstoffkonzentration im \SI{5}{\second}-Intervall gestartet.
Beendet wird die Messung bei einer Sauerstoffkonzentration von rund \SI{8,7}{\milli\gram \per \liter}.

Für eine erneute Messung mit neuer Drehzahl wird ebenfalls wird der absorbierte Sauerstoff mit Stickstoff ausgetrieben und die Messung erneut gestartet.