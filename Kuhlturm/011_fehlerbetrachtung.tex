\newpage
\section{Fehlerbetrachtung}
\label{sec:fehlec|}

Es ist davon auszugehen, dass die größten Fehlerquellen beim Einstellen der Heizleistung des Heizlüfters, des Kühlwasservolumenstromes und dem zeitlich passenden Ablesen der Temperaturen im stationären Zustand des Systems "`Kühlturm"' auftraten. Da sich die Leistung des Heizlüfter nur stufenlos einstellen ließ und die Änderungsrate zwischen der Heizlüfter- Einstellung und der Lufttemperatur sehr war, ließen sich Temperaturen der Luft nur schlecht einstellen. Daher kommt es auch zu den Überschreitungen der Diagrammgrenzen der hx-Diagramme. Weiterhin ließ sich der Kühlwasserstrom weder im Volumen noch in der Temperatur automatisch regeln. Es erforderte daher Feingefühl beim Einstellen des Wasserstromes zusammen mit dem konstant halten der Wassertemperatur. Dieser Umstand könnte durch ein Thermostat behoben werden.\\
 Weiterhin war durch die wechselnden Temperaturen am Computermonitor durch die zuvor genannten Gegebenheiten ein Abschätzen des stationären Zustandes Systems relativ schwierig. Eventuell könnte dieser Punkt behoben werden indem längere Abtastraten der Sensoren dafür sorgen, dass Messwerte in längeren, zeitlichen Abständen aufgenommen werden oder in der Computer Mittelwerte über mehrere Messreihe für bestimmte Zeitintervalle darstellt.

Als weitere Fehlerquelle ist die Dichtigkeit des Kühlturm zu nennen. Dieser zeigte während des Versuches eine geringe Leckage des gekühlten, simulierten Kühlwassers auf und kann somit zum Teil die Messergebnisse verfälschen.

Die Sensoren selbst funktionierten geringfügig Zeitverzögert, zeigten ansonsten für den genutzten Temperaturbereich jedoch keine auffälligen Schwächen auf.

Grundsätzlich liegen die größten Fehlerquellen im Einstellen, der unterschiedlichen Betriebsparameter und nicht in der Messung dieser. Durch prozesstechnische Regelstrukturen könnten diese problematischen Bedingungen eingedämmt werden.