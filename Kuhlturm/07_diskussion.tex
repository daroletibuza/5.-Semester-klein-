\newpage
\section{Diskussion der Ergebnisse}
\label{sec:diskussion}

Ausgehend von den Mittelwerten der berechneten Massen und Volumenströme der durchströmenden Luft erscheinen diese bezüglich des Größenmaßstabes der Versuchsanlage als realistisch. Im einzelnen betrachtet fallen aufgrund der Dichteunterschiede stärkere Schwankungen bei den berechneten Volumenströme als bei den Massenströmen auf. Das hat zur Folge, dass die Volumenströme in diesen Messreihen einen Maximalwert von \SI{204}{\kmeter \per \hour } und einen Minimalwert von \SI{127}{\kmeter\per \hour} erreichten, die Massenströme hingegen einen Maximalwert von \SI{0,06}{\kg \per \second} und einen Minimalwert \SI{0,04}{\kg \per \second}. Auffallend ist hierbei, dass gerade die letzten Datenreihe mit +5K der \mbox{Versuchsreihe 2} sehr niedrig ausfällt. Grund hierfür könnte ein ungünstiges Abpassen der Messwertaufnahme sein. Ohne diese Messreihe werden Minimalwerte von \SI{170}{\kmeter \per \hour} und \SI{0,05}{\kg \per \second}, welche als weniger abweichend zum jeweiligen Mittelwert erscheinen.\\
Ausgehend von in diesem Protokoll nicht aufgeführten Statistiken ergibt mittels \textsc{Grubbs}-Test die letzten Datenreihe mit +5K der Versuchsreihe 2 jedoch keinen statistischen Ausreißer für eine statistische Sicherheit von $p =\SI{95}{\percent}$.\\

In Anbetracht der h-x-Diagramme ist zu verzeichnen, dass bis auf Versuchsreihe 1 ein Teil der Messwerte den Diagrammbereich überschreitet und somit die grafischen Darstellungen für diese Messpunkte als geschätzt anzunehmen ist. Dennoch ist im Vergleich aller drei Diagramme die Tendenz deutlich erkennbar, dass je heißer das Kühlwasser ist, desto höher ist die Enthalpie dieser feuchten Luft und somit der Betrag der Wärme, die vom warmen Kühlwasser übertragen wird. Sprich je wärmer das Kühlwasser ist, desto höher ist der Betrag der übertragenen Wärme vom Kühlwasser an die Luft des Kühlturms und desto größer ist die Temperaturdifferenz zwischen eintretenden und austretenden Kühlwasser. Aber innerhalb jedes der drei Diagramme ist auch zu erkennen, dass mit steigender, einströmender Lufttemperatur die Temperaturdifferenz nicht nur zwischen Kühlwasser und Luft als Triebkraft geringer wird, sondern auch zwischen eintretendem und austretendem Kühlwasser. 
\newpage
Somit lässt sich sagen, dass bei steigender Lufttemperatur und konstanter Wassertemperatur das Kühlwasser ineffizienter gekühlt wird. Zusammengefasst ist dieser Aspekt  auch unter Abbildung \ref{dia:tdiffverlauf} einsehbar. Weiterhin zeigt jede Messreihe eine Verringerung der Temperatur des Kühlwassers (siehe Abb. \ref{dia:temperaturverlauf}), was darauf schließen lässt, dass trotz der zum Teil höheren Temperatur an warmer Luft dennoch eine Abkühlung des Kühlwassers vorliegt. \\
Aus diesen ermittelten Zusammenhängen lässt sich demnach sagen, dass die effizienteste Fahrweise des Kühlturms darin besteht, dass möglichst kalte Luft den Kühlturm durchströmen sollte, um eine höhere Triebkraft der Wärmeübertragung zu bewerkstelligen. Im besten Fall hat die Luft eine geringere Temperatur als das Kühlwasser. Weiterhin arbeitet der Kühlturm am besten, wenn das Kühlwasser vergleichsweise warm in den Kühlturm gegeben wird mit dem Nachteil, dass natürlich auch das zurückgeführte Kühlwasser wärmer ist.