\section{Einleitung und Versuchsziel}
\label{sec:aufgabenstellung}
%In der Aufgabenstellung wird (in eigenen Worten und ganzen Sätzen) formuliert, was das Ziel des 
%Versuches ist.  
%[Beachten Sie die eigentliche Aufgabenstellung in den Versuchsanleitungen sowie die Hinweise zur Auswertung!] 

Im folgenden Versuch wird die Verdunstungskühlung von Wasser anhand eines Kühlturms untersucht. Der Versuchsstand stellt dabei die Kühlung von Kühlwasser mittels warmer Sommerluft dar. Ziel ist es hierbei den Massen- bzw. den Volumenstrom der zu geführten, warmen Luft über Wärme- und Stoffbilanzen zu bestimmen, sowie die Dokumentation der Messwerte in ein \textsc{Molliec|}-Diagramm. Zusätzlich werden ein Fließbild der Anlage und eine Einschätzung über die Effektivität des Prozesses zu gefordert. \\



