\section{Theoretische Grundlagen}
\label{sec:physik}
Grundlage für den Versuch am Kühlturm stellt die Kopplung von Wärme- und Stoffübergang zwischen warmen Kühlwasser und der zugeführten warmen Luft dar. \\

Zu Beginn des Versuches sind dabei die Wärmeverluste durch die Leitung des Kühlwassers ohne Luftstrom zu betrachten. Die Berechnung der Verlustwärme $\dot{Q}_V$ erfolgt in diesem Versuch über die Wärmekapazität des Wassers $c_{P_{\ce{H2O}}}$, der Temperaturdifferenz des Wassers zwischen Eintritt $T_{\alpha, \ce{H2O},V}$ und Austritt $T_{\omega, \ce{H2O},V}$ des Bilanzraumes, sowie dem Massenstrom des Wassers.
\begin{flalign}
	\dot{Q}_V &= \dot{m}_{\ce{H2O}}*c_{P_{\ce{H2O}}}*\left(T_{\alpha, \ce{H2O},V}-T_{\omega, \ce{H2O},V}\right)
\end{flalign}

Nach Ermittlung der Verlustwärme kann nun mit der Gesamtwärme $\dot{Q}_{\text{ges}}$ des jeweiligen laufenden Prozesses mit warmer Luft, die nutzbare Wärme $\dot{Q}_{\text{Nutz}}$ berechnet werden. Wichtig ist dabei, dass die Wassertemperatur und der Massenstrom des Wassers nahezu konstant gehalten werden mussten.

\begin{flalign}
		\dot{Q}_{ges} &= \dot{Q}_{\text{Nutz}}+\dot{Q}_V\\
		\dot{Q}_{\text{Nutz}} &= 	\dot{Q}_{ges} -\dot{Q}_V
\end{flalign}

\begin{flalign}
	\dot{Q}_{\text{Nutz}}  &= \dot{m}_{\ce{H2O}}*c_{P_{\ce{H2O}}}*\left(T_{\alpha, \ce{H2O},\text{ges}}-T_{\omega, \ce{H2O},\text{ges}}\right)-\dot{m}*c_{P_{\ce{H2O}}}*\left(T_{\alpha, \ce{H2O},V} - T_{\omega, \ce{H2O},V}\right) \\
													&= \dot{m}_{\ce{H2O}}*c_{P_{\ce{H2O}}}*\left(T_{\alpha,\ce{H2O},\text{ges}}-T_{\omega, \ce{H2O},\text{ges}}-T_{\alpha, \ce{H2O},V}+T_{\omega, \ce{H2O},V}\right) \\
													&= \dot{m}_{\ce{H2O}}*c_{P_{\ce{H2O}}}*\left(\Delta T_{\ce{H2O},\text{ges}}-\Delta T_{\ce{H2O},V}\right)
\end{flalign} 

Ausgehend von der \textsc{Merkel}'schen Hauptgleichung lässt sich nun eine vereinfachte Energiebilanz zwischen dem Kühlwasser und der warmen Luft aufstellen \cite[S. 1662]{.2013}. Diese vereinfachte Form gilt unter der Annahme, dass die Verdunstungsmenge an Wasser im Vergleich zur Kühlwassermenge vernachlässigbar ist.
\vspace*{-5mm}

\begin{flalign}
	\dot{Q}_L &= \dot{Q}_{\text{Nutz}} \\
	\dot{m}_L * \left(h_{L2}-h_{L1}\right) &= \dot{m}_{\ce{H2O}} *c_{P_{\ce{H2O}}}* \left(T_{\omega, \ce{H2O}}-T_{\alpha, \ce{H2O}}\right) \\
	\dot{m}_L * \left(h_{L2}-h_{L1}\right) &= \dot{m}_{\ce{H2O}} *c_{P_{\ce{H2O}}}*\left(\Delta T_{\ce{H2O},\text{ges}}-\Delta T_{\ce{H2O},V}\right)
\end{flalign}

Umgeformt führt diese Gleichung zum Massenstrom der Luft. Die benötigten spezifischen Enthalpien der feuchten Luft lassen sich mit dem \textsc{Mollier}-Diagramm bestimmen.
 \begin{flalign}
 	\dot{m}_L &= \frac{\dot{Q}_{\text{Nutz}}}{h_{L2}-h_{L1}} = \frac{\dot{m}_{\ce{H2O}} *c_{P_{\ce{H2O}}}*\left(\Delta T_{\ce{H2O},\text{ges}}-\Delta T_{\ce{H2O},V}\right)}{h_{L2}-h_{L1}}
 \end{flalign}

Um den geforderten Volumenstrom zu bestimmen ist zusätzlich eine Berechnung der Dichte für die entsprechenden Messreihen notwendig. In diesem Fall erfolgt die Bestimmung über den Sättigungsdampfdruck des Wassers durch die \textsc{Magnus}-Formel  und der Gaskonstante für feuchte Luft.\\

\textbf{\textsc{Magnus}-Formel}:
\begin{flalign}
	E \left[\si{\pascal}\right]&= \SI{6,112E2}{\pascal}*\mathrm{e}^{\frac{17,62*T_L\left[\si{\celsius}\right]}{\SI{243,12}{\celsius}+T_L\left[\si{\celsius}\right]}}
\end{flalign}

\newpage

\textbf{Allgemeine Gaskonstante der feuchten Luft:}
\begin{flalign}
	R_f \left[\si{\joule \per \kg \per \kelvin}\right]&=\frac{R_t}{1-\varphi*\frac{E}{p}*\left(1-\frac{R_t}{R_d}\right)}
\end{flalign}

\begin{description}
	\item[$p \ldots$] Luftdruck der Umgebung
	\item[$R_t \ldots$] allgemeine Gaskonstante der trockenen Luft $R_t=\SI{287,058}{\joule \per \kg \per \kelvin}$
	\item[$R_d \ldots$] allgemeine Gaskonstante von Wasserdampf $R_d=\SI{461,523}{\joule \per \kg \per \kelvin}$
\end{description}
\vspace*{5mm}

\textbf{Ideale Gasgleichung für Luftdichte:}
\begin{flalign}
	p*V_L &= m_L * R_f*T_L\left[\si{\kelvin}\right]  \tag{$\rho = \frac{m}{V}$}\\
	\rho_L &= \frac{p}{R_f*T_L\left[\si{\kelvin}\right]}
\end{flalign}

Mit der berechneten Luftdichte und dem Massenstrom lässt sich nun berechnen welchen Volumenstrom die durch den Kühlturm strömende Luft hat:
\begin{flalign}
 \dot{V}_L &= \dot{m}_L*\rho_L
\end{flalign}

