\section{Ergebnisse}
\label{sec:ergebnisse}

\subsection*{Destillationsprotokoll}
Nachfolgend ist das aufgenommene Destillationsprotokoll dargestellt (siehe Tab. \ref{tab:dest_pro}).

% Table generated by Excel2LaTeX from sheet 'Daten'
\begin{table}[h!]
	\renewcommand*{\arraystretch}{1.2}
	\centering
	\rowcolors{2}{gray!25}{white}
	\caption{Destillationsprotokoll}
	\label{tab:dest_pro}
		\begin{tabulary}{1.0\textwidth}{C|C|C|C|C|C}
			\hline
			\textbf{Uhrzeit} & $T_{\text{Öl}} \left[\si{\celsius}\right]$ &$T_{\text{Öl}} \left[\si{\celsius}\right]$ & $p \left[\si{\milli \bar}\right]$&$\frac{\text{Tropfen}}{\text{Sekunde}}$&Fraktion\\
			\hline
			\uhr{13}{57}&23&27&15&0&1 (Vorlauf)\\			
			\uhr{13}{59}&31&27&15&0&1\\
			\uhr{14}{01}&54&27&15&0&1\\
			\uhr{14}{03}&63&27&15&0&1\\
			\uhr{14}{05}&76&28&15&0&1\\
			\uhr{14}{07}&88&28&16&0&1\\
			\uhr{14}{09}&95&28&14&0&1\\
			\uhr{14}{11}&100&63&14&1&1\\
			\uhr{14}{13}&100&64&14&2-3&1\\
			\hline
			\uhr{14}{15}&98&64&14&3&2\\
			\uhr{14}{17}&98&64&14&4&2\\
			\uhr{14}{19}&100&64&14&3&2\\
			\uhr{14}{21}&120&64&14&4&2\\
			\uhr{14}{23}&124&64&14&4&2\\
			\uhr{14}{25}&120&64&14&3-4&2\\
			\uhr{14}{27}&119&64&14&3&2\\
			\hline
			\uhr{14}{29}&115&64&14&2-3&3\\
			\uhr{14}{31}&105&64&14&1&3\\
			\uhr{14}{33}&100&62&14&0&3\\
			\hline
	\end{tabulary}
\end{table}%
\FloatBarrier

\subsection*{Massen der Fraktionen}
In der folgenden Tabelle \ref{tab:massen} sind die Massen der Kolben für die einzelnen Fraktionen dargestellt, sowie die sich daraus ergebende Masse der Fraktion.

% Table generated by Excel2LaTeX from sheet 'Daten'
\begin{table}[h!]
	\renewcommand*{\arraystretch}{1.2}
	\centering
	\rowcolors{2}{white}{gray!25}
	\caption{Massen der Destilationsfraktionen}
	\label{tab:massen}
	\resizebox{\textwidth}{!}{
	\begin{tabulary}{1.1\textwidth}{L|C|C|C|C}
		\hline
		\textbf{Fraktion} & \textbf{Fraktion 1} & \textbf{Fraktion 2} & \textbf{Fraktion 3} & \textbf{Fraktion 4}\\
		\hline
		\textbf{Masse Kolben (vorher)} & &&&\\
		\textbf{Massen Kolben (danach)} &&&&\\
		\textbf{Masse Destillat} &&&&\\
		\hline
	\end{tabulary}}
\end{table}
\FloatBarrier

\newpage

\subsection*{Brechungsindices} 
Um die entstandenen Fraktionen in der Zusammensetzung miteinander vergleichen zu können wurden diese mittels Refraktometer untersucht. Dabei wurden der Brechungsindex der jeweiligen Fraktion bestimmt, welche unter Tabelle \ref{tab:brechung} dargestellt sind.

% Table generated by Excel2LaTeX from sheet 'Daten'
\begin{table}[h!]
	\renewcommand*{\arraystretch}{1.2}
	\centering
	\rowcolors{2}{white}{gray!25}
	\caption{Brechungsindices der Fraktionen 1 bis 3}
	\label{tab:brechung}
		\begin{tabulary}{1.0\textwidth}{C|CCCC}
			\hline
			\textbf{Fraktion} & \textbf{1} & \textbf{2}&\textbf{3} &\textbf{4}\\
			\hline
			\textbf{Brechungsindex}&&&&\\
			\hline			
	\end{tabulary}
\end{table}%
\FloatBarrier

\subsection*{Ausbeute}
Die zu Beginn des Versuches abgemessenen \SI{19,8}{\gram} 1,2-Dibrom-1-phenylethan entsprachen einer Stoffmenge von \SI{0,075}{\mol}. Die Masse an entstandenen Phenylacetylen wurde mit \anmerkung{Masse} eingewogen und eine molare Masse von \SI{102}{\gram \per \mole} für das Produkt bestimmt. Aus diesen Angaben wurden die Ausbeute des Produktes in Gleichung \ref{gl:ausbeute} bestimmt. 
\begin{flalign}
	\label{gl:ausbeute}
	\eta 	&= \frac{n_{\text{Phenylacetylen}}}{n_{\text{1,2-Dibrom-1-phenylethan}}} = \frac{m_{\text{Phenylacetylen}}}{n_{\text{1,2-Dibrom-1-phenylethan}}*M_{\text{Phenylacetylen}}}\\[2mm]
	&=	\frac{\SI{0,0}{\gram}}{\SI{0,075}{\mol}*\SI{102}{\gram \per \mole}}\\
	&=\underline{\SI{0,0}{\percent}}
\end{flalign}



