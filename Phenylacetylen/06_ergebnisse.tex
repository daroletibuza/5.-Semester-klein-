\section{Ergebnisse}
\label{sec:ergebnisse}

\subsection*{Destillationsprotokoll}
Nachfolgend ist das aufgenommene Destillationsprotokoll dargestellt (siehe Tab. \ref{tab:dest_pro}).

% Table generated by Excel2LaTeX from sheet 'Daten'
\begin{table}[h!]
	\renewcommand*{\arraystretch}{1.2}
	\centering
	\rowcolors{2}{gray!25}{white}
	\caption{Destillationsprotokoll}
	\label{tab:dest_pro}
		\begin{tabulary}{1.0\textwidth}{C|C|C|C|C}
			\hline
			\textbf{Uhrzeit} & $T_{\text{Kopf}} \left[\si{\celsius}\right]$ & $p \left[\si{\milli \bar}\right]$&$\frac{\text{Tropfen}}{\text{Sekunde}}$&Fraktion\\
			\hline
			\uhr{11}{20}&22&60&0&1\\			
			\uhr{11}{23}&21&50&0&1\\
			\uhr{11}{25}&21&55&0&1\\
			\uhr{11}{26}&24&100&0,5&1\\
			\hline
			\uhr{11}{27}&25&100&1&2\\
			\uhr{11}{28}&27&95&2&2\\
			\uhr{11}{29}&28&95&3&2\\
			\hline
			\uhr{11}{30}&65&70&1&3\\
			\uhr{11}{31}&80&60&0,5&3\\
			\uhr{11}{32}&111&60&0,5&3\\
			\uhr{11}{34}&124&60&1&3\\
			\hline
			\uhr{11}{35}&105&60&1&4\\
			\uhr{11}{36}&60&60&0,5&4\\
	\end{tabulary}
\end{table}%
\FloatBarrier

\subsection*{Massen der Fraktionen}
In der folgenden Tabelle \ref{tab:massen} sind die Massen der Kolben für die einzelnen Fraktionen dargestellt, sowie die sich daraus ergebende Masse der Fraktion.

% Table generated by Excel2LaTeX from sheet 'Daten'
\begin{table}[h!]
	\renewcommand*{\arraystretch}{1.2}
	\centering
	\rowcolors{2}{white}{gray!25}
	\caption{Massen der Destilationsfraktionen}
	\label{tab:massen}
	\resizebox{\textwidth}{!}{
	\begin{tabulary}{1.15\textwidth}{L|C|C|C|C}
		\hline
		\textbf{Fraktion} &\textbf{Fraktion 1} & \textbf{Fraktion 2} & \textbf{Fraktion 3} & \textbf{Fraktion 4}\\
		\hline
		\textbf{Masse Kolben (vorher) in $\left[\si{\gram}\right]$} & 31,22&34,71&30,83&29,28\\
		\textbf{Massen Kolben (danach) in $\left[\si{\gram}\right]$} &32,31&36,08&32,99&29,61\\
		\textbf{Masse Destillat in $\left[\si{\gram}\right]$} &1,09&1,37&2,16&0,33\\
		\hline
	\end{tabulary}}
\end{table}
\FloatBarrier

\subsection*{Brechungsindices} 
Um die entstandenen Fraktionen in der Zusammensetzung miteinander vergleichen zu können wurden diese mittels Refraktometer untersucht. Dabei wurden der Brechungsindex der jeweiligen Fraktion bestimmt, welche unter Tabelle \ref{tab:brechung} dargestellt sind.

% Table generated by Excel2LaTeX from sheet 'Daten'
\begin{table}[h!]
	\renewcommand*{\arraystretch}{1.2}
	\centering
	\rowcolors{2}{white}{gray!25}
	\caption{Brechungsindices der Fraktionen 1 bis 3}
	\label{tab:brechung}
		\begin{tabulary}{1.0\textwidth}{C|CCCC}
			\hline
			\textbf{Fraktion} & \textbf{1} & \textbf{2}&\textbf{3} &\textbf{4}\\
			\hline
			\textbf{Brechungsindex}&1,396&1,406&1,523&1,535\\
			\hline			
	\end{tabulary}
\end{table}%
\FloatBarrier

\subsection*{Ausbeute}
Die zu Beginn des Versuches abgemessenen \SI{21,78}{\gram} 1,2-Dibrom-1-phenylethan entsprachen, unter der Annahme von 100\% Reinheit, einer Stoffmenge von \SI{0,083}{\mol}. Die Masse an entstandenen Phenylacetylen wurde mit \SI{4,95}{\gram} eingewogen und eine molare Masse von \SI{102}{\gram \per \mole} für das Produkt bestimmt. Aus diesen Angaben wurden die Ausbeute des Produktes in Gleichung \ref{gl:ausbeute} bestimmt. 
\begin{flalign}
	\label{gl:ausbeute}
	\eta 	&= \frac{n_{\text{Phenylacetylen}}}{n_{\text{1,2-Dibrom-1-phenylethan}}} = \frac{m_{\text{Phenylacetylen}}}{n_{\text{1,2-Dibrom-1-phenylethan}}*M_{\text{Phenylacetylen}}}\\[2mm]
	&=	\frac{\SI{4,95}{\gram}}{\SI{0,083}{\mol}*\SI{102}{\gram \per \mole}} =\underline{\SI{58,5}{\percent}}
\end{flalign}



