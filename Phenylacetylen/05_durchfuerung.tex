\section{Versuchsdurchführung}
\label{sec:durchfuerung}
\vspace*{-5mm}
\bild{schematischer Versuchsaufbau}{aufbau}{0.5}
\vspace*{-3mm}
\subsection*{Durchführung:}
Begonnen wurde mit der Zugabe von \SI{19,8}{\gram} 1,2-Dibrom-1-phenylethan und \SI{19,6}{\gram} gepulverten Kaliumhydroxid in einen \SI{250}{\milli \liter} Rundkolben. Es folgte gründliches Mischen beider Stoffe im Reaktionskolben. Nach Aufsetzen des Rückflusskühlers erfolgte die langsame Zugabe von \SI{24}{\milli \liter} Ethanol. \anmerkung{Beobachtung der Reaktion}. Nach dem die Reaktion abgeklungen war, wurde der Kolbeninhalt anschließend für \SI{1}{\hour} unter Rückfluss erhitzt.

\subsection*{Isolierung und Reinigung}
Das erhitzte Reaktionsgemisch wurde nun etwas abgekühlt und mit \SI{100}{\milli \liter} Wasser versetzt. Jegliches Salz ging hierbei in Lösung und es entstand ein 2-Phasengemisch. Dieses Gemisch wurde nun in einen Scheidetrichter gegeben und die wässrige Phase abgetrennt. Die abgetrennte, wässrige Phase wird nochmal danach mit \SI{150}{\milli \liter} \textit{tert}-Butylmethylether ebenfalls in einem Scheidetrichter extrahiert.
Beide organischen Phasen, die der erste und die der zweiten Extraktion, wurden daraufhin vereinigt und mit \anmerkung{Menge} Natriumsulfat unter Rühren getrocknet und im Nachgang das Natriumsulfat mit Watte und einem Trichter abfiltriert. Das organische Gemisch wurde dann in einen Rotationsverdampfer gegeben, um den \textit{tert}-Butylmethylether bei \anmerkung{Druck} herauszudestillieren. \\
Der Rückstand wurde infolgedessen in einer Destillationsapparatur mittels fraktionierenden Vakuumdestillation destilliert. Die Kolben wurden hierzu vorher gewogen (siehe \ref{tab:massen}).
Die Destillationsspinne wurde während der Destillation in einem Eisbad gekühlt. Genauere Angaben zur Destillation finden sich unter Tab. \ref{tab:dest_pro}.\\
Im Anschluss an die Destillation wurden die Vorlagekölbchen erneut gemessen (siehe Tab. \ref{tab:massen}) und für die Auswertung die Brechungsindices der einzelnen Fraktionen bestimmt (siehe Tab. \ref{tab:brechung}). 

\subsection*{Entsorgung}
Die wässrig, alkalische Lösung der Extraktion wurde mit \anmerkung{Säure} neutralisiert und im Behälter für Halogenabfälle entsorgt. Das Trockenmittel Natriumsulfat ist im Feststoffabfall entfernt worden und Destillationsrückstände und verunreinigte Fraktionen sind in etwas Aceton gelöst und ebenfalls über den Behälter für Halogenabfälle entsorgt worden. Der abdestillierte \textit{tert}-Butylmethylether wurde recycelt.