\newpage
\section{Diskussion der Ergebnisse}
\label{sec:diskussion}

\anmerkung{Alles überarbeiten wenn Daten vorhanden sind}

%Brechungsindices
Betrachtet man die Brechungsindices der Fraktionen 1 bis 3 lässt sich erkennen, dass sich diese nur geringfügig bis gar nicht voneinander unterscheiden. Diese Tatsache lässt den Schluss zu, dass alle drei Fraktionen zum Großteil dieselbe Zusammensetzung enthalten müssen. Auch die konstante Dampftemperatur von \SI{64}{\celsius} im Destillationsprotokoll (siehe Tab. \ref{tab:dest_pro}) unterstützt diese These.\\
Lediglich Fraktion 1 weicht mit 1,410 vom Brechungsindex der Fraktionen 2 und 3 ab. Das könnte sich jedoch durch den Vorlauf der Destillation erklären lassen, dass in diesem dennoch geringe Verunreinigungen enthalten waren. Dennoch lässt sich sagen, dass alle drei Fraktionen als Hauptbestandteil den Ester in sich tragen.\\

Vergleicht man die Brechungsindices mit einem Literaturwert aus einem chemischen Datenblatt lässt sich ebenfalls feststellen, dass hauptsächlich Essigsäure-n-hexylester entstanden sein wird (siehe Tab. \ref{tab:brechung_lit}).

% Table generated by Excel2LaTeX from sheet 'Daten'
\begin{table}[h!]
	\renewcommand*{\arraystretch}{1.2}
	\centering
	\rowcolors{2}{white}{gray!25}
	\caption{Brechungsindices der Fraktionen 1 bis 4}
	\label{tab:brechung_lit}
	\begin{tabulary}{1.0\textwidth}{C|CCCC|C}
		\hline
		\textbf{Fraktion} & \textbf{1} & \textbf{2}&\textbf{3}&\textbf{4}&Literatur \cite{phenylacetylen_chemspider}\\
		\hline
		\textbf{Brechungsindex}&&&&&1,549\\
		\hline			
	\end{tabulary}
\end{table}%
\FloatBarrier

%Ausbeute
Die Ausbeute von \SI{0,0}{\percent} wird vermutlich hauptsächlich durch Rückstände in den jeweiligen Messgeräten und Apparaturen zu erklären sein. Die Zugabe von Kaliumhydroxid und Ethanol erfolgte im Überschuss, sodass dieser Einfluss ausgeschlossen wird. 