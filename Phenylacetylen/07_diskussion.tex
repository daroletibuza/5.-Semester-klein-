\newpage
\section{Diskussion der Ergebnisse}
\label{sec:diskussion}
%Brechungsindices
Betrachtet man die Brechungsindices der Fraktionen 1 bis 4 lässt sich erkennen, dass sich zusammengefasst di ersten und letzten beiden Fraktionen deutlich voneinander unterscheiden. Diese Tatsache lässt den Schluss zu, dass die ersten zwei Fraktionen 1 und 2 den Vorlauf der Destillation darstellen und die letzten beiden Fraktionen 3 und 4 hauptsächlich das Produkt darstellen. Auch der konstante Druck am Manometer der Pumpe unterstützt diese These für die Fraktionen 3 und 4 (siehe Tab. \ref{tab:dest_pro}).\\
Zudem ist zu erkennen, dass der Brechungsindex mit steigender Fraktionszahl ebenfalls steigt und sich dabei dem Literaturwert von 1,549 annähert \cite{phenylacetylen_chemspider}. Das Spricht für die Tatsache, dass der Anteil an Phenylacetylen mit steigender Fraktionszahl steigt und demnach hauptsächlich zum Ende der Destillation abgetrennt wurde. Die Werte von 1,523 und 1,535 im Vergleich zu 1,549 als Brechungsindices sprechen für die Bildung des Phenylacetylens.

% Table generated by Excel2LaTeX from sheet 'Daten'
\begin{table}[h!]
	\renewcommand*{\arraystretch}{1.2}
	\centering
	\rowcolors{2}{white}{gray!25}
	\caption{Brechungsindices der Fraktionen 1 bis 4}
	\label{tab:brechung_lit}
	\begin{tabulary}{1.0\textwidth}{C|CCCC|C}
		\hline
		\textbf{Fraktion} & \textbf{1} & \textbf{2}&\textbf{3}&\textbf{4}&Literatur \cite{phenylacetylen_chemspider}\\
		\hline
		\textbf{Brechungsindex}&1,396&1,406&1,523&1,535&1,549\\
		\hline			
	\end{tabulary}
\end{table}%
\FloatBarrier

%Ausbeute
Die Ausbeute von \SI{58,5}{\percent} wird vermutlich hauptsächlich durch Rückstände in den jeweiligen Messgeräten und Apparaturen zu erklären sein. Die Zugabe von Kaliumhydroxid und Ethanol erfolgte im Überschuss, sodass dieser Einfluss ausgeschlossen wird. Jedoch sind höchstwahrscheinlich Nebenreaktionen abgelaufen, die nun folgend diskutiert werden.\\

Während der Eliminierung von 1,2-Dibrom-1-phenylethan ist es durchaus möglich, dass die Reaktionsprodukte der Abb. \ref{fig:nebenprodukte} entstanden sein könnten. 
Die Produkte A und B könnten durch unvollständige Eliminierung des Ausgangsstoffes entstanden sein. Sie stellen das Zwischenprodukt auf dem Reaktionsweg zum Phenylacetylen dar (siehe Abb. \ref{fig:mechanismus}). Das Produkt C könnte durch eine Substitutionsreaktion mit dem verwendeten Lösemittel Ethanol entstanden sein. Produkt D könnte die folge einer vollständigen Eliminierung zu Phenylacetylen sein, welches jedoch mit dem Lösemittel interagiert und somit eine Additionsreaktion hervorruft. Das Nebenprodukt E könnte durch eine nukleophile Substitution von Hydroxid-Ionen mit Bromid-Ionen entstanden sein. Das dabei entstehende Enol wird lagert sich aufgrund Keto-Enol-Tautomerie zum Acetophenon um.

\bild{Nebenprodukte der Eliminierung}{nebenprodukte}{0.6}

\newpage

Die Nebenprodukte würden sich zunächst in ihren Siede- und Schmelztemperaturen unterscheiden, aufgrund der unterschiedlichen Struktur in Form von Kettenlängen und Substituenten. Eine massenspektroskopische würde ebenfalls über die molaren Massen die Verbindungen unterscheiden lassen. Die Produkte A und B würden sich in dieser Hinsicht wahrscheinlich jedoch in beiden Fällen nur geringfügig bis gar nicht unterscheiden was Siedetemperatur und maximale molare Masse angeht. Hierfür müsste das Massenspektrum genauer untersucht werden, ob nicht charakteristische "'Bruchstücke"' des jeweiligen Moleküls registriert werden, die im anderen nicht vorkommen. Für Produkt B beispielsweise ein -CHBr-Bruchstück \mbox{(Ende Molekül B)} oder ein -CH2-Bruchstück \mbox{(Ende Molekül A)}.

\vfill

\bibliography{Literatur}
\bibliographystyle{unsrtdin}
\addcontentsline{toc}{section}{Literaturverzeichnis}
