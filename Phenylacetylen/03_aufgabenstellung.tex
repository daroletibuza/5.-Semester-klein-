\section{Einleitung und Versuchsziel}
\label{sec:aufgabenstellung}
%In der Aufgabenstellung wird (in eigenen Worten und ganzen Sätzen) formuliert, was das Ziel des 
%Versuches ist.  
%[Beachten Sie die eigentliche Aufgabenstellung in den Versuchsanleitungen sowie die Hinweise zur Auswertung!] 

Im folgenden Versuch wird aus 1,2-Dibrom-1-phenylethan (siehe Protokoll vom 24.03.2021) mit Hilfe von Kaliumhydroxid, Ethanol und \textit{tert}-Butylmethylether, Phenylacetylen hergestellt. Hauptsächlich werden in diesem Versuch arbeitsmethodische Kenntnisse zur Extraktion und Destillation von Stoffen benötigt. Die Destillationsprodukte werden mittels Refraktometer auf die Brechungsindices untersucht und so die einzelnen Fraktionen miteinander verglichen. Zusätzlich wird diskutiert, wie sich mögliche Nebenprodukte mittels spektroskopischer Daten und einfachen Versuchen unterscheiden lassen.\\
Nachfolgend ist ein möglicher Mechanismus der Eliminierung dargestellt.

\bild{möglicher Mechanismus der Eliminierung zu Phenylacetylen}{mechanismus}{0.9}

