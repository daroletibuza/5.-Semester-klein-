\vfill
\section{Versuchsdurchführung}
\label{sec:durchfuerung}

\subsection*{Durchführung:}
\subsubsection*{Versuchsteil 1: Synthese von Phthalsäurehydrazid}
Der erste Versuchsteil begann mit dem Lösen von \SI{2,5}{\gram} 3-Nitrophthalsäure, \SI{7,5}{\milli \liter} Triethylenglycol und \SI{5}{\milli \liter} 8\%ige Hydrazinhydrat-Lösung in einem \SI{50}{\milli \liter} Zweihalskolben bei 80-\SI{90}{\celsius}. Durch Erwärmen auf 110-\SI{130}{\celsius} wurde das Wasser im Kolben abdestilliert. Nach dieser Destillation wurde der Kolbeninhalt für \SI{2}{\minute} auf 210-\SI{220}{\celsius} erhitzt. Anschließend wurde der Inhalt des Reaktionskolben auf unter \SI{100}{\celsius} abgekühlt und mit \SI{30}{\milli \liter} siedendes, heißes Wasser versetzt. Danach ist die Lösung auf auf Raumtemperatur abgekühlt worden und in Folge dessen mit Eis für \SI{30}{\minute} auskristallisiert.\\
Mit Hilfe eines \textsc{Büchner}-Trichters wurde dann das beigefarbene Phthalsäurehydrazid abgesaugt. Die Ausbeute des feuchten Produktes beträgt \anmerkung{\SI{0,0}{\gram}}.

\vfill

\subsubsection*{Versuchsteil 2: Synthese des Luminol}
Das feuchte Produkt aus Versuchsteil 1 wurde zurück in den Kolben gegeben und mit \SI{12,5}{\milli \liter} 10\%ige Natronlauge unter Rühren aufgelöst. Es entstand eine dunkel-rotbraune Lösung, welche daraufhin portionsweise mit \anmerkung{4,2*Ausbeute\SI{0,0}{\gram}} Natriumdithionit versetzt wurde. Es folgte ein 15 minütiges Sieden unter Rückfluss mit darauffolgendem Abkühlen. Danach wurden \SI{5}{\milli \liter} Essigsäure hinzugegeben und es fiel Luminol als hellgelber Feststoff aus. Dieser Niederschlag wurde mit Hilfe einer Zentrifuge von der wässrigen Phase abgetrennt, in dem diese Phase nach dem Zentrifugieren dekantiert wurde. Es erfolgte eine erneute Aufschlämmung des Luminol mit etwas Wasser und die darauffolgende Zentrifugation und Dekantieren der wässrigen Phase. Schlussendlich wurde der Feststoff auf ein Uhrglas überführt und bei \SI{110}{\celsius} im Trockenschrank getrocknet. Die Ausbeute nach dem Trocknen betrug \anmerkung{\SI{0,0}{\gram} (maximal 12 mmol) laut Anleitung rund 0,7g}.

\subsection*{Versuchsteil 3: Chemolumineszenz des Luminol}
In diesem Versuchsteil soll die Chemolumineszenz des Luminols untersucht werden. Dafür wurden \SI{120}{\milli \gram} des synthetisierten Luminol zusammen mit \SI{5}{\milli \liter} 10\%iger Natronlauge in \SI{45}{\milli \liter} Wasser gelöst. Parallel dazu wurde die lumineszenzauslösende Lösung aus \SI{160}{\milli \liter} Wasser, \SI{2}{\milli \liter} 3\%iger Wasserstoffperoxid-Lösung und \SI{20}{\milli \liter} Kaliumhexacyanoferrat(III)-Lösung hergestellt.
Nun wurden \SI{25}{\milli \liter} der Luminol-Lösung mit Wasser auf ein Volumen von \SI{200}{\milli \liter} verdünnt.\\
Zur Beobachtung der Chemolumineszenz wechselte man in einen dunklen Raum. In diesem wurden in einem \SI{500}{\milli \liter} Erlenmeyerkolben die beiden Lösungen gemischt. Es ist ein blaues Leuchten der Lösung wahrzunehmen, welches sich durch Schwenken oder Zugabe von Natronlauge verstärkt.

\subsection*{Entsorgung}
Alles Laborgeräte welche Hydrazin-Reste enthalten konnten, wurden mit Wasserstoffperoxid versetzt bis keine Gasentwicklung mehr zu beobachten war. Die Entsorgung aller Waschphasen, auch die der wässrigen Phase beim Zentrifugieren, erfolgte über die wässrigen Abfälle.