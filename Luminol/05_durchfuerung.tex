\section{Versuchsdurchführung}
\label{sec:durchfuerung}

\vspace*{-7mm}
\bild{Versuchsaufbau}{aufbau}{0.6}
\vspace*{-7mm}

\subsection*{Durchführung:}
\subsubsection*{Versuchsteil 1: Synthese von Phthalsäurehydrazid}
Der erste Versuchsteil begann mit dem Lösen von \SI{2,5}{\gram} 3-Nitrophthalsäure, \SI{7,5}{\milli \liter} Triethylenglycol und \SI{5}{\milli \liter} 8\%ige Hydrazinhydrat-Lösung in einem \SI{50}{\milli \liter} Zweihalskolben der aufgebauten Destillationsapparatur bei 80-\SI{90}{\celsius}. Hierfür wurde der Heizpilz auf Stufe 10 und der Rührer auf \SI{500}{\rpm} eingestellt. Nach \SI{15}{\minute} erreichte die Temperatur im Kolben \SI{50}{\celsius} und die Stoffe hatten sich bereits zu einer orange(-braunen) Lösung umgesetzt.\\
Durch weiteres Erwärmen für \SI{15}{\minute} wurde das Wasser im Kolben abdestilliert. Bei \SI{110}{\celsius} sind erstmals Tropfen am Dampfthermometer zu erkennen, sowie eine Änderung der Dampftemperatur von Raumtemperatur auf \SI{60}{\celsius}.  Bei \SI{118}{\celsius} begann die Lösung erstmals zu sieden und die Dampftemperatur erhöhte sich weiter auf \SI{88}{\celsius}. Ab einer Kolbentemperatur von \SI{126}{\celsius} fiel zwar der erste Tropfen des abdestillierten Wassers/Hydrazins in den Auffangkolben, jedoch kondensierte ein Großteil des Dampfes noch vor der Destillationsbrücke. Aus diesem Grund wurde das "`U"' der Destillationsbrücke mit Alufolie umwickelt um ein solch frühzeitiges Auskondensieren zu vermeiden.\\
Daraufhin stieg die Dampftemperatur auf konstante \SI{98}{\celsius} und weitere Tropfen folgten im Auffangkolben. \SI{5}{\minute} später erreichte der Inhalt des Reaktionskolbens eines Temperatur von \SI{210}{\celsius} und die Dampftemperatur sank auf \SI{80}{\celsius}. Durch verstellen der Hebebühne wurde der Kontakt zwischen Heizpilz und Reaktionskolben reguliert und somit auch dessen Temperatur. Diese Temperatur wurde für \SI{2}{\min} auf \SI{210}{\celsius} bis \SI{220}{\celsius} gehalten.\\
Anschließend wurde der Inhalt des Reaktionskolben auf unter \SI{100}{\celsius} abgekühlt und mit \SI{30}{\milli \liter} siedendes, heißes Wasser versetzt. Danach ist die Lösung auf auf Raumtemperatur abgekühlt worden und in Folge dessen mit Eis für \SI{30}{\minute} mit bis zu einer Temperatur von \SI{1}{\celsius} auskristallisiert. Während des kompletten Abkühlungsvorganges ist ein Ausfallen eines feinen gelben Niederschlages zu beobachten, durch welchen die Lösung gelblich-trüb erscheint.\\
Mit Hilfe eines \textsc{Büchner}-Trichters und eines zuvor ausgewogenen Filterpapiers ($m_{FP}=\SI{0,2}{\gram}$) wurde dann das gelbliche Phthalsäurehydrazid abgesaugt. Filterkuchen und Filterpapier wurden nach diesem Vorgang zusammen gewogen ($m_{FP+P}=\SI{1,6}{\gram}$).
Die Ausbeute des feuchten Produktes beträgt demnach \SI{1,4}{\gram}.

\subsubsection*{Versuchsteil 2: Synthese des Luminol}
Das feuchte Produkt aus Versuchsteil 1 wurde zurück in den Kolben gegeben und mit \SI{12,5}{\milli \liter} 10\%ige Natronlauge unter Rühren aufgelöst. Es entstand eine dunkel-rotbraune Lösung, welche daraufhin portionsweise mit \SI{5,88}{\gram} Natriumdithionit versetzt wurde. Die Lösung färbte sich daraufhin trüb-orange. Es folgte ein 15 minütiges Sieden bei \SI{100}{\celsius} unter Rückfluss mit darauffolgendem Abkühlen auf \SI{60}{\celsius}. Während dieser Vorgang abläuft wird bereits Versuchsteil 3 vorbereitet und durchgeführt. Nach \SI{20}{\minute} wurden \SI{5}{\milli \liter} Essigsäure hinzugegeben und es fiel Luminol als hellgelber Feststoff aus. Dieser Niederschlag wurde mit Hilfe einer Zentrifuge von der wässrigen Phase abgetrennt, in dem diese Phase nach dem Zentrifugieren dekantiert wurde. Es erfolgte eine erneute Aufschlämmung des Luminols mit etwas Wasser und die darauffolgende Zentrifugation und das Dekantieren der wässrigen Phase. Schlussendlich wurde der Feststoff zum Teil auf eine Abdampfschale überführt. Nach dem Hinweis, dass sich der Feststoff trocken besser aus den Zentrifugengläsern lösen lässt, wurden diese zusammen mit der Abdampfschale in den Trockenschrank gestellt und getrocknet.
Die Ausbeute nach dem Trocknen betrug \SI{1,88}{\gram}. Das übersteigt die laut Anleitung zu erwartende Ausbeute von \SI{0,7}{\gram}

\subsection*{Versuchsteil 3: Chemolumineszenz des Luminols}
In diesem Versuchsteil soll die Chemolumineszenz des Luminols untersucht werden. Zeitlich erfolgten Durchführung und Vorbereitung während Versuchsteil 2. Dafür wurden \SI{120}{\milli \gram} eines studentischen Präparates an Luminol zusammen mit \SI{5}{\milli \liter} 10\%iger Natronlauge in \SI{45}{\milli \liter} Wasser gelöst. Parallel dazu wurde die lumineszenzauslösende Lösung aus \SI{160}{\milli \liter} Wasser, \SI{2}{\milli \liter} 3\%iger Wasserstoffperoxid-Lösung und \SI{20}{\milli \liter} Kaliumhexacyanoferrat(III)-Lösung hergestellt.
Nun wurden \SI{25}{\milli \liter} der Luminol-Lösung mit Wasser auf ein Volumen von \SI{200}{\milli \liter} verdünnt.
\newpage
Zur Beobachtung der Chemolumineszenz wechselte man in einen dunklen Raum. In diesem wurden in einem \SI{500}{\milli \liter} Erlenmeyerkolben die beiden Lösungen gemischt. Es ist ein blaues Leuchten der Lösung durch die Chemolumineszenz wahrgenommen worden.

\subsection*{Entsorgung}
Alles Laborgeräte welche Hydrazin-Reste enthalten konnten, wurden mit Wasserstoffperoxid versetzt bis keine Gasentwicklung mehr zu beobachten war. Die Entsorgung aller Waschphasen, auch die der wässrigen Phase beim Zentrifugieren, erfolgte über die wässrigen Abfälle.