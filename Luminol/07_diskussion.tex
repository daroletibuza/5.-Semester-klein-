\section{Diskussion der Ergebnisse}
\label{sec:diskussion}

In Betrachtung der offensichtlichsten Beobachtung dieses Versuches - das Leuchten der Luminol-Lösung nach Zugabe der Kaliumhexacyanoferrat(III)-Lösung - wird davon ausgegangen, dass die Chemolumineszenz durch Zersetzung von Luminol stattgefunden hat. \\
Das Kaliumhexacyanoferrat(III) wirkt hierbei in Form einer Eisenverbindung, ähnlich wie Hämoglobin beim forensischen Blutnachweis, dabei als Katalysator für die Zersetzung des Wasserstoffperoxids. Dabei entstehen Sauerstoff und Wasser. Die Hydroxid-Ionen, der Natronlauge entfernen währenddessen als Basen die Protonen aus dem Stickstoffatom und bilden eine Struktur im Luminol, welche zusammen mit dem reinem Sauerstoff der Wasserstoffperoxidzersetzungzu einem instabilen Peroxid reagiert. Es erfolgt die sofortige Bildung von Aminophthalsäure im angeregten Zustand Zusammen mit der Freisetzung von Stickstoff. Wenn sich die Aminophthalsäure in den Grundzustand entspannt wird dabei überschüssige Energie als sichtbares Licht freigesetzt. Dieser Vorgang wird als Chemolumineszenz bezeichnet \cite{Sabnis.2009}.\\

Vergleicht man den gemessenen Schmelzpunkt mit den Schmelzbereichen der Literatur, so erkennt man \anmerkung{hier weiter schreiben.}\\

Die Ausbeute von \anmerkung{\SI{0,0}{\percent}} wird vermutlich hauptsächlich durch Rückstände in den jeweiligen Messgeräten und Apparaturen zu erklären sein.