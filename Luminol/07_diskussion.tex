\section{Diskussion der Ergebnisse}
\label{sec:diskussion}

In Betrachtung der offensichtlichsten Beobachtung dieses Versuches - das Leuchten der Luminol-Lösung nach Zugabe der Kaliumhexacyanoferrat(III)-Lösung - wird davon ausgegangen, dass die Chemolumineszenz durch Zersetzung von Luminol stattgefunden hat. \\
Das Kaliumhexacyanoferrat(III) wirkt hierbei in Form einer Eisenverbindung, ähnlich wie Hämoglobin beim forensischen Blutnachweis, dabei als Katalysator für die Zersetzung des Wasserstoffperoxids. Dabei entstehen Sauerstoff und Wasser. Die Hydroxid-Ionen, der Natronlauge entfernen währenddessen als Basen die Protonen aus dem Stickstoffatom und bilden eine Struktur im Luminol, welche zusammen mit dem reinem Sauerstoff der Wasserstoffperoxidzersetzungzu einem instabilen Peroxid reagiert. Es erfolgt die sofortige Bildung von Aminophthalsäure im angeregten Zustand Zusammen mit der Freisetzung von Stickstoff. Wenn sich die Aminophthalsäure in den Grundzustand entspannt wird dabei überschüssige Energie als sichtbares Licht freigesetzt. Dieser Vorgang wird als Chemolumineszenz bezeichnet \cite{Sabnis.2009}.

Vergleicht man den gemessenen Schmelzpunkt mit den Schmelzbereichen der Literatur, so ist eine Temperaturdifferenz von über \SI{100}{K} zu erkennen. Diese hohe Abweichung spricht für eine starke Verunreinigung des Luminols. Vermutlich beruht diese auf unzureichendes Aufschlämmen und Zentrifugieren in der Durchführung unter Versuchsteil 2, wodurch Rückstände an Natriumsulfat im Produkt verbleiben könnten. Gerade die Tatsachen, dass vereinzelt weiße Kristalle im Produkt zu erkennen sind und reines Natriumsulfat eine hohe Schmelztemperatur (\SI{884}{\celsius}) hat, sprechen für diese Verunreinigung. \cite{Sitzmann.2009}\\ 
Als Handlungsempfehlung könnte in der Praktikumsanleitung ein weiterer Aufschlämm- und Zentrifugiervorgang ergänzt werden.

Die Ausbeute von \SI{88,5}{\celsius} aufgrund der Schmelzpunktuntersuchung keine sinnvolle Angabe, da die Masse des synthetisierten Luminols durch Natriumsulfat verunreinigt ist. Da die Ausbeute jedoch auch nicht den Wert von \SI{100}{\percent} übersteigt st davon auszugehen, dass sich das Luminol trotz starker Verunreinigung auch nicht vollständig im Produkt befindet. Erklären lässt sich dies durch Rückstände von Luminol an Apparaturen und Messgeräten.