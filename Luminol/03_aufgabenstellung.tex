\section{Einleitung und Versuchsziel}
\label{sec:aufgabenstellung}
%In der Aufgabenstellung wird (in eigenen Worten und ganzen Sätzen) formuliert, was das Ziel des 
%Versuches ist.  
%[Beachten Sie die eigentliche Aufgabenstellung in den Versuchsanleitungen sowie die Hinweise zur Auswertung!] 

Im folgenden Versuch wird Luminol dargestellt und auf seine Chemolumineszenzeigenschaften untersucht. Im ersten Teilversuch wird aus 3-Nitrophthalsäure, Triethylenglycol, Hydrazinhydrat, Natriumdithionit, Natronlauge und Essigsäure, Luminol hergestellt. Im zweiten Teilversuch wird mit Hilfe von Natronlauge und Kaliumhexacyanoferrat das Luminol auf seine Chemolumineszenz untersucht. Hauptsächlich werden in diesem Versuch arbeitsmethodische Kenntnisse im Erhitzen mittels Rückflusskühlung und Zentrifugieren benötigt. Das Luminol wird mit Hilfe der Schmelzpunktbestimmung auf seine Reinheit geprüft.\\
Nachfolgend sind die Mechanismen, der Luminolsynthese und der Lichtabgabe durch Zersetzung von Luminol unter Einwirkung von Eisen(III)-Ionen dargestellt.

\begin{figure}[h!]
	\begin{minipage}[c]{0.45\linewidth}
		\includegraphics[width=\linewidth]{01_Luminol_Synthese.png}
		\caption{Mechanismus der Luminolsynthese aus 3-Nitrophtalsäure}
	\end{minipage}
	\hfill
	\begin{minipage}[c]{0.45\linewidth}
		\includegraphics[width=\linewidth]{02_Luminol_Lumineszenz}
		\caption{Chemolumineszenzreaktion von Luminol}
	\end{minipage}%
\end{figure}
\FloatBarrier
