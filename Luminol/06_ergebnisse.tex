\section{Ergebnisse}
\label{sec:ergebnisse}

\subsection*{Ausbeute}
Die zu Beginn des Versuches abgemessenen \SI{2,5}{\gram} 3-Nitrophthalsäure und \SI{5}{\milli\liter} 8\%ige Hydrazinhydrat-Lösung entsprachen Stoffmengen von jeweils \SI{12}{\milli \mol}. Die Masse an synthetisierten Luminol wurde trocken mit \anmerkung{\SI{0,0}{\gram}} eingewogen und eine molare Masse von \SI{177}{\gram \per \mole} bestimmt. Aus diesen Angaben wurden die Ausbeute der Luminol-Synthese in Gleichung \ref{gl:ausbeute} bestimmt.\\
\anmerkung{Berechnung überarbeiten !} 
\begin{flalign}
	\label{gl:ausbeute}
	\eta 	&= \frac{n_{\text{Luminol}}}{n_{\text{3-Nitrophthalsäure}}} = \frac{m_{\text{Luminol}}}{n_{\text{3-Nitrophthalsäure}}*M_{\text{Luminol}}}\\[2mm]
	&=	\frac{\SI{0,0}{\gram}}{\SI{12}{\milli \mol}*\SI{177}{\gram \per \mole}}\\
	&=\underline{\SI{0,0}{\percent}}
\end{flalign}

\subsection*{Schmelzpunktbestimmung}
Über den Schmelzpunkt ließ sich bestimmen, wie rein das Syntheseprodukt vorliegt. Für das synthetisierte Luminol wurde eine Schmelztemperatur von \anmerkung{\SI{0,0}{\celsius}} bestimmt. Im Vergleich dazu gab eine Literaturrecherche für reines Luminol eine Schmelztemperatur von 329-\SI{332}{\celsius} bzw. 319-\SI{320}{\celsius} an \cite{Luminol_Rompp}.


\subsection*{Beobachtung der Chemolumineszenz}
Durch Mischung der peroxidischen Kaliumhexacyanoferrat(III)-Lösung mit der Luminol-Lösung entsteht eine hellblau leuchtende Lösung.


